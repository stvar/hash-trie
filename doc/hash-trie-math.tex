% Copyright (C) 2016  Stefan Vargyas
% 
% This file is part of Hash-Trie.
% 
% Hash-Trie is free software: you can redistribute it and/or modify
% it under the terms of the GNU General Public License as published by
% the Free Software Foundation, either version 3 of the License, or
% (at your option) any later version.
% 
% Hash-Trie is distributed in the hope that it will be useful,
% but WITHOUT ANY WARRANTY; without even the implied warranty of
% MERCHANTABILITY or FITNESS FOR A PARTICULAR PURPOSE.  See the
% GNU General Public License for more details.
% 
% You should have received a copy of the GNU General Public License
% along with Hash-Trie.  If not, see <http://www.gnu.org/licenses/>.

\documentclass[a4paper,9pt,leqno]{article}
\usepackage[hang]{footmisc}
\usepackage[fleqn]{amsmath}
\usepackage{listings}
\usepackage{relsize}
\usepackage{tocloft}
\usepackage{quoting}
\usepackage{amsfonts}
\usepackage{amssymb}
\usepackage{amsthm}
\usepackage{suffix}
\usepackage{ifpdf}
\usepackage{calc}
\usepackage{url}

\ifpdf
\pdfinfo{
/Title(The Hash-Trie of Knuth & Liang -- Addendum)
/Author(Stefan Vargyas, stvar@yahoo.com)
/Creator(LaTeX)
}
\fi

\newcommand{\vargyas}{\c{S}tefan Vargyas}

\renewcommand{\ttdefault}{pcr}
\renewcommand{\familydefault}{\sfdefault}

\addtolength{\textheight}{3.4cm}
\addtolength{\topmargin}{-2.1cm}
\addtolength{\textwidth}{3cm}
\addtolength{\evensidemargin}{-1.5cm}
\addtolength{\oddsidemargin}{-1.5cm}

\renewcommand{\=}{\protect\nobreakdash-\hspace{0pt}}
\renewcommand{\~}{\protect\nobreakdash--\hspace{0pt}}

\setlength{\parindent}{0pt}

\newcommand{\plusplus}{\textbf{\raisebox{1pt}{++}}}
\newcommand{\cplusplus}{\textbf{C}\plusplus}
\newcommand{\Hashtrie}{Hash\=\!Trie}
\newcommand{\HashTrie}{\textbf{\Hashtrie}}

\newcommand{\ie}{i.e.}
\newcommand{\code}[1]{{\tt{#1}}}

\makeatletter
% stev: alter the definition of key 'numbers':
\lst@Key{showskiplines}f[f]{%
	\lstKV@SetIf{#1}\lst@ifshowskiplines}
\let\lst@ifskiplines\iffalse
\newcommand\skipnumbering[1]{%
	\setcounter{lstnumber}{\numexpr#1-1\relax}%
	\let\lst@ifskiplines\iftrue
	\\\lst@PlaceNumber
	\let\lst@ifskiplines\iffalse
	\lst@ifshowskiplines...\fi
}
\def\lst@renderskiplinessymbol{{%
	\relsize{1}%
	\lst@skiplinessymbol
	\hspace{-.11em}}%
}
\def\lst@numberorsymbol{
	\lst@ifskiplines
	\makebox[\widthof{\lst@renderskiplinessymbol}+.11em][l]{%
		\lst@renderskiplinessymbol}%
	\else
	\thelstnumber
	\fi
}
\def\lst@defskiplinessymbol{%
	\textbf{.\hspace{.11em}.\hspace{.11em}.}}
\def\lst@skiplinessymbol{}
\lst@Key{skiplinessymbol}{%
	\lst@defskiplinessymbol}{%
	\def\lst@skiplinessymbol{#1}}
\lst@Key{numbers}{none}{%
	\let\lst@PlaceNumber\@empty
	\lstKV@SwitchCases{#1}{%
		none&\\%
		left&\def\lst@PlaceNumber{%
			\llap{%
				\normalfont
				\lst@numberstyle{\lst@numberorsymbol}%
				\kern\lst@numbersep}}\\%
		right&\def\lst@PlaceNumber{%
			\rlap{%
				\normalfont
				\kern\linewidth \kern\lst@numbersep
				\lst@numberstyle{\lst@numberorsymbol}}}%
	}{%
		\PackageError{Listings}{Numbers #1 unknown}\@ehc
	}%
}
\makeatother

% http://www.bollchen.de/blog/2011/04/good-looking-line-breaks-with-the-listings-package/
\newcommand{\shellprebreak}{%
	\raisebox{0ex}[0ex][0ex]{%
		\ensuremath{\hookleftarrow}}}
\newcommand{\shellpostbreak}{%
	\raisebox{0ex}[0ex][0ex]{%
		\ensuremath{\hookrightarrow}\hspace{2pt}}}
\newcommand{\srcprebreak}{%
	\shellprebreak}
\newcommand{\srcpostbreak}{%
	\shellpostbreak}

\newlength\shelllinewidth
\setlength\shelllinewidth{1.35\linewidth}

\newcommand{\shellskipamount}{%
	\baselineskip}
\newcommand{\shellinnerskipamount}{%
	2\parskip}
\newcommand{\shellafterskipamount}{%
	\parskip}
\newcommand{\shellafterskip}{%
	\addvspace{-\shellafterskipamount}}%
\newcommand{\srcskipamount}{%
	\shellskipamount}
\newcommand{\srcafterskipamount}{%
	\shellafterskipamount}
\newcommand{\srcafterskip}{%
	\addvspace{-\srcafterskipamount}}%

\lstdefinestyle{srclistingstyle}{%
	numbers=left,
	numbersep=1.3em,
	numberstyle=\tiny,
	showstringspaces=false,
	basicstyle=\smaller\ttfamily,
	breakindent=0pt,breakautoindent=false,
	breaklines=true,breakatwhitespace=true,
	linewidth=\shelllinewidth,lineskip=1.05pt,
	abovecaptionskip=0pt,
	belowcaptionskip=0pt,
	%!!!aboveskip=\srcskipamount,
	%!!!belowskip=\srcskipamount,
	postbreak=\srcpostbreak,
	prebreak=\srcprebreak,
	emphstyle=\textbf,
}
\lstdefinestyle{cpplistingstyle}{%
	style=srclistingstyle,
	language=[GNU]C++,
	tabsize=4,
	emph={
		alignas,
		alignof,
		auto,
		decltype,
		nullptr,
		constexpr,
		final,
		override,
		static_assert,
		noexcept,
		__func__,
		__FILE__,
		__LINE__,
		__VA_ARGS__,
	},
}
\lstnewenvironment{hashtrielisting}[1][1]{%
	\lstset{%
		literate={!!!INT}{$\ INT\!\!:$}4,
		style=cpplistingstyle,
		showskiplines=true,
		lineskip=-.82pt,
		firstnumber=#1,
		escapechar=@,
	}%
}{%
	%!!!\srcafterskip
}

\swapnumbers
\theoremstyle{plain}
\newtheorem{fact}{Fact}
\newtheorem{axioms}[fact]{Axioms}
\newtheorem{prop}[fact]{Proposition}
\theoremstyle{definition}
\newtheorem{defn}[fact]{Definition}
\newtheorem{notn}[fact]{Notation}
\theoremstyle{remark}
\newtheorem{rem}[fact]{Remark}

\newcommand{\floor}[1]{\lfloor#1\rfloor}
\newcommand{\ceil}[1]{\lceil#1\rceil}
\newcommand{\rimpl}{\:\:\Longleftarrow\:\:}
\newcommand{\impll}{\:\:\Longrightarrow\:\:}
\newcommand{\impl}{\impll}%!!!{\:\:\Rightarrow\:\:}
\newcommand{\iffl}{\:\:\Longleftrightarrow\:\:}
\newcommand{\qimpl}{``$\Rightarrow$''}
\newcommand{\qrimpl}{``$\Leftarrow$''}

\newcommand{\parref}[1]{(\ref{#1})}
\newcommand{\by}[1]{{#1}}
\newcommand{\bydef}{\by{def}}
\newcommand{\bynot}{\by{not}}
\newcommand{\byhyp}{\by{hyp}}
\newcommand{\symby}[2]{\stackrel{#1}{{#2}}}
\newcommand{\symbyrm}[2]{\symby{{\rm\scriptstyle{#1}}}{#2}}
\newcommand{\symbyit}[2]{\symby{{\mit\scriptstyle{#1}}}{#2}}
\newcommand{\implby}[1]{\symbyrm{#1}{\impll}}
\newcommand{\implbyref}[1]{\implby{\parref{#1}}}
\newcommand{\implbyhyp}{\implby{\byhyp}}
\newcommand{\iffby}[1]{\symbyrm{#1}{\iffl}}
\newcommand{\iffbydef}{\iffby{\bydef}}
\newcommand{\iffbyref}[1]{\iffby{\parref{#1}}}
\newcommand{\eqby}[1]{\symbyrm{#1}{=}}
\newcommand{\eqbydef}{\eqby{\bydef}}
\newcommand{\eqbyref}[1]{\eqby{\parref{#1}}}
\newcommand{\eqbyhyp}{\eqby{\byhyp}}
\newcommand{\eqbynot}{\eqby{\bynot}}
\newcommand{\leby}[1]{\symbyrm{#1}{\le}}
\newcommand{\lebyref}[1]{\leby{\parref{#1}}}
\newcommand{\lebyhyp}{\leby{\byhyp}}
\newcommand{\geby}[1]{\symbyrm{#1}{\geq}}
\newcommand{\gebyref}[1]{\geby{\parref{#1}}}
\newcommand{\gebyhyp}{\geby{\byhyp}}
\newcommand{\gt}{>}
\newcommand{\gtby}[1]{\symbyrm{#1}{\gt}}
\newcommand{\gtbyref}[1]{\gtby{\parref{#1}}}
\newcommand{\gtbyhyp}{\gtby{\byhyp}}
\newcommand{\lt}{<}
\newcommand{\ltby}[1]{\symbyrm{#1}{\lt}}
\newcommand{\ltbyref}[1]{\ltby{\parref{#1}}}
\newcommand{\ltbyhyp}{\ltby{\byhyp}}
\newcommand{\inby}[1]{\symbyrm{#1}{\in}}
\newcommand{\inbyref}[1]{\inby{\parref{#1}}}
\newcommand{\sseteqby}[1]{\symbyrm{#1}{\subseteq}}
\newcommand{\sseteqbydef}{\sseteqby{\bydef}}
\newcommand{\sseteqbyref}[1]{\sseteqby{\parref{#1}}}
\newcommand{\sseteqbyhyp}{\sseteqby{\byhyp}}
\newcommand{\sseteqbynot}{\sseteqby{\bynot}}

\newcommand{\mc}{,\:}
\newcommand{\mcbs}{\mc\;\;}
\newcommand{\mcss}{\mc\;}

\newcommand{\lp}{\left(}
\newcommand{\lb}{\left[}
\newcommand{\lc}{\left\{}
\newcommand{\rp}{\right)}
\newcommand{\rb}{\right]}
\newcommand{\rc}{\right\}}
\newcommand{\la}{\left\langle}
\newcommand{\ra}{\right\rangle}

\newcommand\lnref[1]{line \lnref*{#1}}
\WithSuffix
\newcommand\lnref*[1]{\#\ref{#1}}
\newcommand\lnrefs[2]{lines \lnrefs*{#1}{#2}}
\WithSuffix
\newcommand\lnrefs*[2]{\lnref*{#1}\~\lnref*{#2}}

\newcommand{\factref}[1]{fact \ref{#1}}
\newcommand{\Factref}[1]{Fact \ref{#1}}

\newcommand\Nat{\mathbb{N}}
\WithSuffix
\newcommand\Nat*{\mathbb{N}^*}

\newcommand\Int{\mathbb{Z}}
\WithSuffix
\newcommand\Int*{\mathbb{Z}^*}

\newcommand\Real{\mathbb{R}}
\WithSuffix
\newcommand\Real*{\mathbb{R}^*}

\renewcommand{\contentsname}{Table of Contents}
\renewcommand{\cftsecleader}{\cftdotfill{\cftdotsep}}

\title{The \Hashtrie\ of Knuth \& Liang\\---\,Addendum\,---}
\author{\larger\vargyas\medskip\\ \tt\textbf{stvar@yahoo.com}}
\date{Nov 28, 2015}

% http://tex.stackexchange.com/questions/236781/align-the-footnote-markers-to-the-right-in-footnotes/236802#236802
\makeatletter
\patchcmd{\@makefntext}
	{\@makefnmark\hss}
	{\hss\@makefnmark\hspace{4pt}}
	{}{}
\makeatletter

\begin{document}
\maketitle
\tableofcontents

\section{Introduction}

The purpose of the following sections is to provide mathematical proofs for
the claims made at the end of section 2.3, \emph{The implementation of
\code{HashTrie<>} Class Template}, on page 21 of the technical report
\emph{The \Hashtrie\ of Knuth \& Liang: A C\plusplus11 Implementation},
\textbf{hash-trie-impl.pdf}:

\begin{quoting}
The replacement of expression ``\code{h + tolerance - mod\_x}''  with the
expression ``\code{add(h, tolerance2) - mod\_x}'' had to be done, because,
as it is easily provable, under cetain conditions, the subexpression
``\code{h + tolerance}'' is exceeding the upper bound \code{trie\_size}
of the boxed integer \code{pointer\_t}. It is worthy of notice the readily
provable fact that both assignments to \code{last\_h} on lines
\ref{cpp-last-h-assign-1} and \ref{cpp-last-h-assign-2} are correct:
each of the expression on the right side of these assignments do not
exceed upon evaluation the bounds of \code{pointer\_t}.
\end{quoting}

The claim made by the first phrase is restated and proved by
\factref{add-h-tolerance}; the claim of the second phrase -- by
\factref{last-h-bounds}.
Alongside these two facts, the section containing them establishes by proof
a few more facts which concern the inner workings of the core algorithm
of class \code{HashTrie<>}.

\section{Mathematical Evaluations}

\begin{defn}[Defining parameters]
\begin{align}
& trie\_size \in \Nat,   \label{trie-size}\\
%
& max\_letter \in \Nat*. \label{max-letter}\\
%
& tolerance \in \Nat.    \label{tolerance}
\end{align}
\end{defn}

\begin{defn}[Dependent constants]
\begin{align}
mod\_x & \eqbydef trie\_size - 2 \cdot max\_letter, \label{mod-x}\\
%
alpha & \eqbydef \ceil{0.61803 \cdot mod\_x},       \label{alpha}\\
%
max\_h & \eqbydef mod\_x + max\_letter,             \label{max-h}\\
%
max\_x & \eqbydef mod\_x - alpha.                   \label{max-x}
\end{align}
\end{defn}

\begin{axioms}[Initiating assertions]
\begin{align}
& alpha > 0,          \label{A1}\\
%
& tolerance > 0,      \label{A2}\\
%
& mod\_x > 0,         \label{A3}\\
%
& max\_h > tolerance, \label{A4}\\
%
& mod\_x > tolerance, \label{A5}\\
%
& mod\_x > alpha.     \label{A6}
\end{align}
\end{axioms}

\begin{prop}
\begin{align}
& trie\_size \gt 0,                                        \label{trie-size-gt-0}\\
%
& max\_h \gt mod\_x,                                       \label{max-h-gt-mod-x}\\
%
& \parref{A5} \impl \parref{A4},                           \label{a5-impl-a4}\\
%
& tolerance \lt trie\_size,                                \label{tol-lt-trie-size}\\
%
& max\_h = trie\_size - max\_letter,                       \label{max-h-2}\\
%
& max\_h \lt trie\_size,                                   \label{max-h-lt-trie-size}\\
%
& \parref{A4} \iff max\_letter + tolerance \lt trie\_size. \label{a4-tol-max-letter}
\end{align}
\end{prop}

\begin{proof}
For \parref{trie-size-gt-0}: apply \parref{A3} and \parref{mod-x}.

For \parref{max-h-gt-mod-x}: apply \parref{max-h} and \parref{max-letter}.

For \parref{a5-impl-a4}: apply \parref{max-h-gt-mod-x}.

For \parref{tol-lt-trie-size}: $tolerance \ltbyref{A5} mod\_x \eqbyref{mod-x}
trie\_size - 2 \cdot max\_letter \ltbyref{max-letter} trie\_size$.

For \parref{max-h-2}: apply \parref{max-h} and \parref{mod-x}.

For \parref{max-h-lt-trie-size}: apply \parref{max-h-2} and \parref{max-letter}.

For \parref{a4-tol-max-letter}: $\text{\parref{A4}}
\iffbyref{max-h-2} trie\_size - max\_letter \gt tolerance
\iff max\_letter + tolerance \lt trie\_size$.
\end{proof}

\begin{defn}[The \emph{floor} function]
\begin{align}
& \floor{x} \eqbydef \max \mathcal{M}(x) \in \Int\text{, for $x \in \Real$,}
                                                           \label{floor-def}\\
\intertext{where}
& \mathcal{M}(x) \eqbydef \lc k \in \Int \mid k \le x \rc. \label{floor-em}
\end{align}
\end{defn}

\begin{prop}[Basic properties of ``$\floor{\cdot}$'']
\begin{align}
& \floor{x} \le x\text{, for $x \in \Real$,} \label{floor-le}\\
%
& n = \floor{n}\text{, for $n \in \Int$,}    \label{floor-eq}\\
%
& \floor{x} \le \floor{y} \iff \mathcal{M}(x) \subseteq \mathcal{M}(y)\text{, for $x, y \in \Real$,}
                                             \label{floor-em-inc}\\
%
& x \le y \impl \floor{x} \le \floor{y}\text{, for $x, y \in \Real$,}
                                             \label{floor-inc}\\
%
& n \le x \iff n \le \floor{x}\text{, for $n \in \Int$ and $x \in \Real$,}
                                             \label{floor-int}\\
%
& n \le x \lt n + 1 \iff \floor{x} = n\text{, for $n \in \Int$ and $x \in \Real$,}
                                             \label{floor-int-2}\\
%
& a \bmod b = a - \floor{a / b} \cdot b\text{, for $a \in \Int$ and $b \in \Nat*$.}
                                             \label{floor-mod}
\end{align}
\end{prop}

\begin{proof}
For \parref{floor-le}: $x \in \Real \implbyref{floor-def} \floor{x} \in \mathcal{M}(x)
\implbyref{floor-em} \floor{x} \le x$.

For \parref{floor-eq}: by \parref{floor-le} $\floor{n} \le n;\;
n \le n \in \Int \implbyref{floor-em} n \in \mathcal{M}(n)
\impl n \le \max\mathcal{M}(n) \eqbyref{floor-def} \floor{n}$.

For \parref{floor-em-inc}: \qimpl: $k \in \mathcal{M}(x)
\impl k \le \max \mathcal{M}(x) 
\eqbyref{floor-def} \floor{x} \lebyhyp \floor{y}
\inbyref{floor-def} \mathcal{M}(y)
\implbyref{floor-em} k \in \mathcal{M}(y)$; \qrimpl: $\floor{x}
\inbyref{floor-def} \mathcal{M}(x) \sseteqbyhyp \mathcal{M}(y)
\impl \floor{x} \in \mathcal{M}(y)
\implbyref{floor-def} \floor{x} \le \floor{y}$.

For \parref{floor-inc}: $x \le y
\implbyref{floor-em} \mathcal{M}(x) \subseteq \mathcal{M}(y)
\implbyref{floor-em-inc} \floor{x} \le \floor{y}$.

For \parref{floor-int}: \qimpl: $n \le x
\implbyref{floor-inc} n \eqbyref{floor-eq} \floor{n} \le \floor{x}$;
\qrimpl: $n \le \floor{x} \inbyref{floor-def} \mathcal{M}(x)
\implbyref{floor-em} n \le x$.

For \parref{floor-int-2}: $n \le x \iffbyref{floor-int} n \le \floor{x}$;
$x \lt n + 1 \iffbyref{floor-int} \floor{x} \lt n + 1 \iff \floor{x} \le n$.

For \parref{floor-mod}: firstly remark the uniqueness part of the
\emph{Euclidean division theorem}: for $x, y \in \Int$, with $y \gt 0$: if
exists $q' \mc q'' \mc r' \mc r'' \in \Int$ such that $x = q' \cdot y + r'$, 
$x = q'' \cdot y + r''$, with $0 \le r' \mc r'' \lt y$, then $q' = q''$ and $r' = r''$.
The proof of this goes easily: suppose $r' \gt r''$. Then $r' - r'' = y \cdot
(q'' - q')$; follows that $q'' - q' \gt 0$; thus $q'' - q' \ge 1$;
hence $r' - r'' \ge y$, which contradicts
$r' - r'' \lt y \rimpl 0 \le r' \mc r'' \lt y$.

Now, this uniqueness property leads to \parref{floor-mod} if
shown that $0 \le a - b \cdot \floor{a / b} \lt b$. Indeed the relation holds true:
$n = \floor{a / b} \iffbyref{floor-int-2} n \le a / b \lt n + 1
\iff b \cdot n \le a \lt b \cdot (n + 1)
\iff 0 \le a - b \cdot n \lt b$.
\end{proof}

\begin{prop}
For $a \mc b \mc x \in \Int$:
\begin{align}
& 0 \le a \lt b \impl a \bmod b = a,                       \label{a-mod-b-eq-a}\\
%
& 0 \lt b \le a \lt 2 \cdot b \impl a \bmod b = a - b,     \label{a-mod-b-eq-a-minus-b}\\
%
& 0 \le x \lt max\_x \impl (x + alpha) \bmod mod\_x = x + alpha,
                                                           \label{x-alpha-mod-x-lt}\\
%
& max\_x \le x \lt mod\_x \impl (x + alpha) \bmod mod\_x = x - max\_x.
                                                           \label{x-alpha-mod-x-ge}
\end{align}
\end{prop}

\begin{proof}
For \parref{a-mod-b-eq-a}: $0 \le a \lt b \impl 0 \le a/b \lt 1
\implbyref{floor-int-2} \floor{a/b} = 0 \implbyref{floor-mod} a \bmod b = a$.

For \parref{a-mod-b-eq-a-minus-b}: $0 \lt b \le a \lt 2 \cdot b \impl 1 \le a/b \lt 2
\implbyref{floor-int-2} \floor{a/b} = 1 \implbyref{floor-mod} a \bmod b = a - b$.

For \parref{x-alpha-mod-x-lt}: $x \lt max\_x \iffbyref{max-x} x + alpha \lt mod\_x$;
$x \ge 0 \implbyref{A1} x + alpha \gt 0$; then \parref{a-mod-b-eq-a} implies
\parref{x-alpha-mod-x-lt}.

For \parref{x-alpha-mod-x-ge}: $x \ge max\_x \iffbyref{max-x} x + alpha \ge mod\_x$;
$alpha \ltbyref{A6} mod\_x\text{ and }x \ltbyhyp mod\_x \impl x + alpha \lt 2 \cdot mod\_x$;
the latter two consequents along with \parref{A3} conclude to
$(x + alpha) \bmod mod\_x \eqbyref{a-mod-b-eq-a-minus-b} x + alpha - mod\_x
\eqbyref{max-x} x - max\_x$. 
\end{proof}

\begin{notn}
For $h \in \Nat*$ let $h' \eqbydef h + tolerance \in \Nat*$ and
$h'' \eqbydef h' - mod\_x \in \Int$.
\end{notn}

\begin{prop}
\begin{align}
& h \le max\_h - tolerance \iff h' \le max\_h,          \label{h-le-max-h-minus-tol}\\
%
& h \ge max\_letter + 1 \impl h' \gt max\_letter + 1,   \label{h-max-letter-plus-1}\\
%
& h \le max\_h \iff h'' \le max\_letter + tolerance,    \label{h-plus-tol-minus-mod-x-le}\\
%
& h \gt max\_h - tolerance \iff h'' \gt max\_letter,    \label{h-plus-tol-minus-mod-x-gt}\\
%
& max\_h - tolerance \lt h \le max\_h \iff 
max\_letter \lt h'' \le max\_letter + tolerance,        \label{h-plus-tol-minus-mod-x}\\
%
& max\_letter + tolerance \le max\_h,                   \label{max-letter-plus-tol-lt-max-h}\\
%
& max\_letter \lt h \le max\_h \iff 0 \le max\_h - h \lt mod\_x,
                                                        \label{max-h-minus-h}\\
%
& max\_h + tolerance \gt trie\_size \iff tolerance \gt max\_letter.
                                                        \label{max-h-plus-tol}
\end{align}
\end{prop}

\begin{proof}
For \parref{h-le-max-h-minus-tol}: $h \le max\_h - tolerance
\iff h' = h + tolerance \le max\_h$.

For \parref{h-max-letter-plus-1}: $h' = h + tolerance
\gebyhyp max\_letter + 1 + tolerance
\gtbyref{A2} max\_letter + 1$.

For \parref{h-plus-tol-minus-mod-x-le}: $h \le max\_h
\iff h'' = h + tolerance - mod\_x \le max\_h - mod\_x + tolerance
\eqbyref{max-h} max\_letter + tolerance$.

For \parref{h-plus-tol-minus-mod-x-gt}: $h \gt max\_h - tolerance
\iff h'' = h + tolerance - mod\_x \gt max\_h - mod\_x
\eqbyref{max-h} max\_letter$.

For \parref{h-plus-tol-minus-mod-x}: apply \parref{h-plus-tol-minus-mod-x-le}
and \parref{h-plus-tol-minus-mod-x-gt}.

For \parref{max-letter-plus-tol-lt-max-h}: $max\_letter + tolerance
\lebyref{A5} max\_letter + mod\_x
\eqbyref{max-h} max\_h$. 

For \parref{max-h-minus-h}: $max\_letter \lt h \le max\_h
\iff 0 \le max\_h - h \lt max\_h - max\_letter
\eqbyref{max-h} mod\_x$.

For \parref{max-h-plus-tol}: $max\_h + tolerance \gt trie\_size
\iff tolerance \gt trie\_size - max\_h
\eqbyref{max-h-2} max\_letter$. 
\end{proof}

\begin{prop}
\begin{align}
& max\_letter + 1 \le h \le max\_h - tolerance \impl max\_letter + 1 \le h' \le max\_h,
                                                             \label{hi-bounds}\\
%
& max\_h - tolerance \lt h \le max\_h \impl max\_letter + 1 \le h'' \le max\_h.
                                                             \label{hii-bounds}
\end{align}
\end{prop}

\begin{proof}
For \parref{hi-bounds}: by \parref{h-max-letter-plus-1} $h' \gt max\_letter + 1$,
therefore $h' \ge max\_letter + 1$. By \parref{h-le-max-h-minus-tol} $h' \le max\_h$.

For \parref{hii-bounds}: by \parref{h-plus-tol-minus-mod-x-gt} $h'' \gt max\_letter$,
therefore $h'' \ge max\_letter + 1$. By \parref{h-plus-tol-minus-mod-x-le} and
\parref{max-letter-plus-tol-lt-max-h} $h'' \le max\_h$.
\end{proof}

\section{Applications to \HashTrie}

\begin{fact}
The definitions
\parref{trie-size},
\parref{max-letter} and
\parref{tolerance}
correspond to lines
\lnref*{cpp-trie-size},
\lnref*{cpp-max-letter} and
\lnref*{cpp-tolerance}
respectively of the \cplusplus\ implementation.
\end{fact}

\begin{fact}
The definitions
\parref{mod-x},
\parref{alpha},
\parref{max-h} and
\parref{max-x}
correspond to lines
\lnref*{cpp-mod-x},
\lnref*{cpp-alpha},
\lnref*{cpp-max-h} and
\lnref*{cpp-max-x}
respectively of the \cplusplus\ implementation.
\end{fact}

\begin{fact}
The axioms \parref{A1}\~\parref{A6} correspond one by one to the
\code{CXX\_ASSERT} \lnrefs{cpp-assert-begin}{cpp-assert-end}.
\end{fact}

\begin{fact}
The assignment on \lnref{cpp-tol2-init} is correct.
\end{fact}

\begin{proof}
Due to the definition of \code{pointer\_t}, the statement on
\lnref{cpp-tol2-init} is correct if and only if \code{tolerance} is
greater or equal than \code{0} and less or equal than \code{trie\_size}
(these are the limiting bounds of \code{pointer\_t}). From \parref{A2}
and \parref{tol-lt-trie-size} results that the \lnref{cpp-tol2-init}
is indeed correct.
\end{proof}

\begin{fact}\label{x-sequence}
The variable \code{x} declared, initialized and maintained on lines
\lnref*{cpp-x},
\lnref*{cpp-x-init} and respectively
\lnrefs*{cpp-x-begin}{cpp-x-end} is
iterating correctly the elements of the sequence $\lp x_n\rp_{n \in \Nat}$,
where $x_n \eqbydef (alpha \cdot n) \bmod mod\_x\text{ for }n \in \Nat$.
\end{fact}

\begin{proof}
Proceed by induction on $n \in \Nat$. For $n = 0$ the statement made above
holds, since the \lnref{cpp-x-init} is showing that the initial
value of \code{x} is \code{0}. Now suppose that prior to executing
\lnref{cpp-x-begin} \code{x} has the value of $x_n$ for some $n \in \Nat$.
In view of the relations \parref{x-alpha-mod-x-lt} and \parref{x-alpha-mod-x-ge},
upon the execution of \lnrefs{cpp-x-begin}{cpp-x-end}, \code{x} becomes
\code{(x + alpha) \emph{mod} mod\_x}. Taking into account that, by the
definition of sequence $\lp x_n\rp_{n \in \Nat}$,
$x_{n+1} = (x_n + alpha) \bmod mod\_x$, indeed \code{x} is $x_{n+1}$ after
the \lnref{cpp-x-end}.
\end{proof}

\begin{fact}\label{post-h-assign-assert}
The assertion stated within the comment on \lnref{cpp-post-h-assign-assert}
is correct.
\end{fact}

\begin{proof}
Need to prove that upon executing \lnref{cpp-post-h-assign-assert},
$max\_letter \lt h \le trie\_size - max\_letter \eqbyref{max-h-2} max\_h$:
$\text{\Factref{x-sequence}}
\impl 0 \le x \lt mod\_x \eqbyref{max-h} max\_h - max\_letter
\iff -1 \lt x \le max\_h - max\_letter - 1
\iff max\_letter \lt h
\eqby{\lnref*{cpp-post-h-assign-assert}} x + max\_letter + 1 \le max\_h$.
\end{proof}

\begin{fact}\label{add-h-tolerance}
Under certain conditions, the result of evaluating the expression
\code{add(h, tolerance2)} on \lnref{cpp-last-h-assign-1} exceeds the value of
\code{trie\_size} for some \code{h}.
\end{fact}

\begin{proof}
By \factref{post-h-assign-assert}: $max\_letter \lt h \le max\_h$. If
let $h \eqbydef max\_h$, then, under the condition that $tolerance \gt max\_letter$,
\parref{max-h-plus-tol} shows that $h + tolerance \gt trie\_size$ indeed.
\end{proof}

\begin{rem}
The fact above indicates that the expression \code{h + tolerance} wouldn't have
been a proper choice of coding the \lnref{cpp-last-h-assign-1}: in the case of
$h + tolerance \gt trie\_size$, the evaluation of the expression
\code{h + tolerance} would have caused the program to halt abruptly (assuming
that the configuration parameter \code{CONFIG\_HASH\_TRIE\_STRICT\_TYPES} was
\code{\#define}d at compile-time).
\end{rem}

\begin{fact}\label{last-h-bounds}
The assignments to variable \code{last\_h} on lines \lnref*{cpp-last-h-assign-1} and
\lnref*{cpp-last-h-assign-2} are both correct. Upon the execution of either of them,
$max\_letter + 1 \le last\_h \le max\_h$.
\end{fact}

\begin{proof}
The statements on lines \lnref*{cpp-last-h-assign-1} and \lnref*{cpp-last-h-assign-2}
are correct if and only if each of the expression on the right side of the
respective assignments evaluates to an integer not exceeding the bounds of
type \code{pointer\_t}.
By the \factref{post-h-assign-assert}, before executing each of the two lines:
$max\_letter \lt h \le max\_h$.
Now, for the case of \lnref{cpp-last-h-assign-1} apply \parref{hii-bounds}
(from \lnref{cpp-h-gt-max-h-minus-tol}: $h \gt max\_h - tolerance$) and,
respectively, for the case of \lnref{cpp-last-h-assign-2} apply \parref{hi-bounds}
(from \lnref{cpp-h-gt-max-h-minus-tol}: $h \le max\_h - tolerance$).
Both give that $max\_letter + 1 \le last\_h \le max\_h$.
Consequently, the bounds of \code{pointer\_t} are respected: by
\parref{max-letter} and \parref{max-h-lt-trie-size}, the previous
double inequality yields: $0 \lt last\_h \lt trie\_size$.
\end{proof}

\begin{fact}
The inner loops of method \code{HashTrie<>::find} (not displayed by the listing
below) that are based on \code{h} computed by \lnrefs{cpp-next-h-begin}{cpp-next-h-end}
are finite.
\end{fact}

\begin{proof}
By the \factref{post-h-assign-assert}, each of these loops start iterating
with an \code{h} satisfying $max\_letter + 1 \le h \le max\_h$. The
\lnrefs{cpp-next-h-begin}{cpp-next-h-end} show that \code{h} is incremented
circularly within the boundaries $max\_letter + 1$ and $max\_h$.
By the \factref{last-h-bounds}, $max\_letter + 1 \le last\_h \le max\_h$ on each
execution of \lnrefs{cpp-next-h-begin}{cpp-next-h-end}, \ie\ \code{last\_h}
lies between the same boundaries as \code{h}. The implementation code also
shows (not seen below, though) that \code{last\_h} is an invariant of each of
these loops. Consequently, \code{h} has to meet \code{last\_h} upon
a finitely many succesive calls of the lambda function
\code{compute\_the\_next\_trial\_header\_location}.
This leads the lambda function to return \code{false} -- thus terminating the
iterations.
\end{proof}

\appendix
\section{\protect\cplusplus\ Implementation Excerpts}
%
% $ sha1sum hash-trie.cpp
% 7ba8b1459f0198c93b78c9e1a96faae7f8512824  hash-trie.cpp
% $ hash-trie-source --hash-trie --tex -t -r 1994-2010 -r 2012-2018 -r 2021 -r 2023-2206 -r 2209-2225 -r 2227-2231 -r 2233-2235 -r 2237-2242 -r 2254-2272 -l 2019:cpp-trie-size -l 2020:cpp-tolerance -l 2022:cpp-max-letter -l 2208:cpp-x -l 2243:cpp-alpha -l 2244:cpp-mod-x -l 2245:cpp-max-h -l 2246:cpp-max-x -l 2248:cpp-assert-begin -l 2253:cpp-assert-end
%
\begin{hashtrielisting}[1983]
template<
	typename C = char,
	template<typename> class T = char_traits_t,
	typename S = size_traits_t>
class HashTrie :
	private T<C>,
	private S
{
public:
	typedef S size_traits_t;
	typedef T<C> char_traits_t;@\skipnumbering{2011}@
private:@\skipnumbering{2019}@
	using size_traits_t::trie_size;@\label{cpp-trie-size}@
	using size_traits_t::tolerance;@\label{cpp-tolerance}@@\skipnumbering{2022}@
	using char_traits_t::max_letter;@\label{cpp-max-letter}@@\skipnumbering{2207}@
	// x_n = (alpha * n) % mod_x
	pointer_t x;@\label{cpp-x}@@\skipnumbering{2226}@
	pointer_t find(const char_t*);@\skipnumbering{2232}@
	static constexpr size_t make_alpha(size_t trie_size, size_t max_letter)@\skipnumbering{2236}@
	{ return std::ceil(0.61803 * (trie_size - 2 * max_letter)); }@\skipnumbering{2243}@
	static constexpr size_t alpha = make_alpha(trie_size, max_letter);@\label{cpp-alpha}@
	static constexpr size_t mod_x = trie_size - 2 * max_letter;@\label{cpp-mod-x}@
	static constexpr size_t max_h = mod_x + max_letter;@\label{cpp-max-h}@
	static constexpr size_t max_x = mod_x - alpha;@\label{cpp-max-x}@

	CXX_ASSERT(alpha > 0);@\label{cpp-assert-begin}@
	CXX_ASSERT(tolerance > 0);
	CXX_ASSERT(trie_size > 2 * max_letter);
	CXX_ASSERT(max_h > tolerance);
	CXX_ASSERT(mod_x > tolerance);
	CXX_ASSERT(mod_x > alpha);@\label{cpp-assert-end}@@\skipnumbering{2273}@
};
\end{hashtrielisting}
%
% $ hash-trie-source --hash-trie-func=HashTrie --tex -t -r 2281-2294 -l 2295:cpp-x-init
%
\begin{hashtrielisting}[2275]
template<
	typename C,
	template<typename> class T,
	typename S>
HashTrie<C, T, S>::HashTrie()
{@\skipnumbering{2295}@
	x = 0;@\label{cpp-x-init}@
}
\end{hashtrielisting}
%
% $ hash-trie-source --hash-trie-func=find --tex -t -r 2306-2421 -r 2430-2453 -r 2460-2482 -r 2484-2499 -r 2503-2509 -r 2516-2521 -r 2530-2618 -l 2422:cpp-tol2-init -l 2454:cpp-x-begin -l 2457:cpp-x-end -l 2459:cpp-post-h-assign-assert -l 2483:cpp-h-gt-max-h-minus-tol -l 2500:cpp-last-h-assign-1 -l 2510:cpp-last-h-assign-2 -l 2522:cpp-next-h-begin -l 2528:cpp-next-h-end
%
\begin{hashtrielisting}[2298]
template<
	typename C,
	template<typename> class T,
	typename S>
typename
	HashTrie<C, T, S>::pointer_t
	HashTrie<C, T, S>::find(const char_t* str)
{@\skipnumbering{2422}@
	const pointer_t tolerance2 = tolerance;@\label{cpp-tol2-init}@
	// trial header location
	pointer_t h;
	// the final one to try
	pointer_t last_h; //!!!INT int last_h;

	const auto get_set_for_computing_header_locations = [&]() {
		// 24. Get set for computing header locations@\skipnumbering{2454}@
		if (x >= max_x)@\label{cpp-x-begin}@
			x -= max_x;
		else
			x += alpha;@\label{cpp-x-end}@

		h = x + max_letter + 1; // now max_letter < h <= trie_size - max_letter@\label{cpp-post-h-assign-assert}@@\skipnumbering{2483}@
		if (h > max_h - tolerance) {@\label{cpp-h-gt-max-h-minus-tol}@@\skipnumbering{2500}@
			last_h = add(h, tolerance2) - mod_x;@\label{cpp-last-h-assign-1}@
		}
		else {@\skipnumbering{2510}@
			last_h = h + tolerance;@\label{cpp-last-h-assign-2}@
		}
	};

	const auto compute_the_next_trial_header_location = [&]() {
		// 25. Compute the next trial header location h, or abort find@\skipnumbering{2522}@
		if (h == last_h)@\label{cpp-next-h-begin}@
			return false;
		if (h == max_h)
			h = max_letter + 1;
		else
			h ++;
		return true;@\label{cpp-next-h-end}@
	};@\skipnumbering{2619}@
}
\end{hashtrielisting}
%
\let\thefootnote\relax%
\footnotetext{%
\renewcommand{\Hashtrie}{Hash-Trie}%
Copyright \copyright\ 2016\hspace{1em}\textbf{\vargyas}

This file is part of \HashTrie.

\HashTrie\ is free software: you can redistribute it and/or modify
it under the terms of the GNU General Public License as published by
the Free Software Foundation, either version 3 of the License, or
(at your option) any later version.

\HashTrie\ is distributed in the hope that it will be useful,
but WITHOUT ANY WARRANTY; without even the implied warranty of
MERCHANTABILITY or FITNESS FOR A PARTICULAR PURPOSE.  See the
GNU General Public License for more details.

You should have received a copy of the GNU General Public License
along with \HashTrie.  If not, see \url{http://www.gnu.org/licenses/}.}
%
\end{document}


