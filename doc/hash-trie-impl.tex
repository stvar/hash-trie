% Copyright (C) 2016  Stefan Vargyas
% 
% This file is part of Hash-Trie.
% 
% Hash-Trie is free software: you can redistribute it and/or modify
% it under the terms of the GNU General Public License as published by
% the Free Software Foundation, either version 3 of the License, or
% (at your option) any later version.
% 
% Hash-Trie is distributed in the hope that it will be useful,
% but WITHOUT ANY WARRANTY; without even the implied warranty of
% MERCHANTABILITY or FITNESS FOR A PARTICULAR PURPOSE.  See the
% GNU General Public License for more details.
% 
% You should have received a copy of the GNU General Public License
% along with Hash-Trie.  If not, see <http://www.gnu.org/licenses/>.

\documentclass[a4paper,11pt]{article}
\usepackage[nottoc,numbib]{tocbibind}
\usepackage[hang]{footmisc}
\usepackage{cellspace}
\usepackage{tabularx}
\usepackage{environ}
\usepackage{booktabs}
\usepackage{varioref}
\usepackage{hyphenat}
\usepackage{setspace}
\usepackage{listings}
\usepackage{dashrule}
\usepackage{enumitem}
\usepackage{titlesec}
\usepackage{quoting}
\usepackage{nameref}
\usepackage{upquote}
\usepackage{tocloft}
\usepackage{amsmath}
\usepackage{relsize}
\usepackage{fnbreak}
\usepackage{suffix}
\usepackage{ifthen}
\usepackage{ifpdf}
\usepackage{calc}
\usepackage{cite}
\usepackage{url}

\ifpdf
\pdfinfo{
/Title(The Hash-Trie of Knuth & Liang: A C++11 Implementation)
/Author(Stefan Vargyas, stvar@yahoo.com)
/Creator(LaTeX)
}
\fi

\newcommand{\vargyas}{\c{S}tefan Vargyas}

\renewcommand{\ttdefault}{pcr}
\renewcommand{\familydefault}{\sfdefault}
\usepackage{mathastext}

\addtolength{\textheight}{3.4cm}
\addtolength{\topmargin}{-2.1cm}
\addtolength{\textwidth}{2.75cm}
\addtolength{\evensidemargin}{-2cm}
\addtolength{\oddsidemargin}{-2cm}

\setlength{\parindent}{0pt}
\setlength{\parskip}{1.5\medskipamount}

\renewcommand{\=}{\protect\nobreakdash-\hspace{0pt}}
\renewcommand{\~}{\protect\nobreakdash--\hspace{0pt}}

\def\UrlFont{\bf}

\newcommand{\gnu}{GNU}
\newcommand{\gnulinux}{\gnu/Linux}
\newcommand{\plusplus}{\textbf{\raisebox{1pt}{++}}}
\newcommand{\cplain}{$C$}
\newcommand{\cplusplus}{$C\plusplus$}
\newcommand{\Cplusplus}{$\textbf{C\protect\plusplus}$}
\newcommand{\pascal}{\code{Pascal}}
\newcommand{\Hashtrie}{Hash\=\!Trie}
\newcommand{\HashTrie}{\textbf{\Hashtrie}}
\WithSuffix
\newcommand\HashTrie*{Hash\hspace{-.105em}Trie}
\newcommand{\hashtrie}{{\tt\textbf{hash-trie}}}
\WithSuffix
\newcommand\hashtrie*{\emph{hash\=trie}}

\newcommand{\ie}{i.e.}
\newcommand{\eg}{e.g.}
\newcommand{\opt}[1]{\nohyphens{\tt{`#1\textquotesingle}}}
\newcommand{\code}[1]{{\tt{#1}}}
\WithSuffix
\newcommand\code*[1]{\mbox{\code{#1}}}
\newcommand{\prog}[1]{{\tt\textbf{#1}}}
\newcommand{\module}[1]{\code{#1}}
\newcommand{\txt}[1]{{\textit{#1.txt}}}
\newcommand{\pg}[1]{p.~#1}
\newcommand{\pp}[2]{pp.~#1\~#2}
\newcommand{\pag}[1]{page~#1}
\newcommand{\pages}[2]{pages~#1\~#2}
\newcommand{\lno}[1]{\mbox{$\#$\hspace{1pt}#1}}
\newcommand{\para}[1]{\mbox{\S\hspace{1pt}#1}}
\newcommand{\paras}[2]{\mbox{\S\hspace{1pt}#1--#2}}
\newcommand{\parapag}[3][]{%
	\ifthenelse{\not\equal{#1}{}}{%
		\para{#2}, \emph{#1}, on \pag{#3}%
	}{%
		\para{#2} on \pag{#3}%
	}%
}
\newcommand{\parapg}[2]{\para{#1}, \pg{#2}}
\newcommand{\paraspag}[3]{\paras{#1}{#2} on \pag{#3}}
\newcommand{\paraspages}[4]{\paras{#1}{#2} on \pages{#3}{#4}}
\newcommand{\paraspp}[4]{\paras{#1}{#2} on \pp{#3}{#4}}
\newcommand{\shaonenamed}{\code{sha1}\=named}
\newcommand{\compiletime}{compile\=time}
\newcommand{\runtime}{run\=time}
\newcommand{\onetoone}{one\=to\=one}

\newcommand{\bitemlabel}{$\bullet$}
\newlength\bitemlabelwidth
\newlength\bitemleftmargin
\newcommand{\bitemsetparams}[1]{%
	\setlength\bitemlabelwidth{\labelindent}%
	\addtolength\bitemlabelwidth{\widthof{#1}}%
	\setlength\bitemleftmargin{\bitemlabelwidth}%
	\addtolength\bitemleftmargin{\labelsep}%
	\addtolength\bitemleftmargin{.5\leftmargin}%
}
\bitemsetparams{\bitemlabel}
\newlist{bitemize}{itemize}{9}
\setlist[bitemize]{
	align=left,
	parsep=.5\parsep,%!!!0pt,
	partopsep=0pt,
	topsep=0pt,
	itemsep=0pt,
	itemindent=0pt,
	labelindent=0pt,
	label=\bitemlabel,
	labelwidth=\bitemlabelwidth,
	leftmargin=\bitemleftmargin,
}

\makeatletter
\newcommand\ProcessPreInit@stev{%
	\gdef\lst@frameL@##1{\hdashrule[.5ex][c]{2pt}{1pt}{1pt}}%!!!SHELLVERTLINE
}
% stev: hook into listings (see listings.dtx)
\lst@AddToHook{PreInit}{\ProcessPreInit@stev}

% stev: alter the definition of key 'numbers':
\lst@Key{showskiplines}f[f]{%
	\lstKV@SetIf{#1}\lst@ifshowskiplines}
\let\lst@ifskiplines\iffalse
\newcommand\skipnumbering[1]{%
	\setcounter{lstnumber}{\numexpr#1-1\relax}%
	\let\lst@ifskiplines\iftrue
	\\\lst@PlaceNumber
	\let\lst@ifskiplines\iffalse
	\lst@ifshowskiplines...\fi
}
\def\lst@renderskiplinessymbol{{%
	\relsize{1}%
	\lst@skiplinessymbol
	\hspace{-.11em}}%
}
\def\lst@numberorsymbol{
	\lst@ifskiplines
	\makebox[\widthof{\lst@renderskiplinessymbol}+.11em][l]{%
		\lst@renderskiplinessymbol}%
	\else
	\thelstnumber
	\fi
}
\def\lst@defskiplinessymbol{%
	\textbf{.\hspace{.11em}.\hspace{.11em}.}}
\def\lst@skiplinessymbol{}
\lst@Key{skiplinessymbol}{%
	\lst@defskiplinessymbol}{%
	\def\lst@skiplinessymbol{#1}}
\lst@Key{numbers}{none}{%
	\let\lst@PlaceNumber\@empty
	\lstKV@SwitchCases{#1}{%
		none&\\%
		left&\def\lst@PlaceNumber{%
			\llap{%
				\normalfont
				\lst@numberstyle{\lst@numberorsymbol}%
				\kern\lst@numbersep}}\\%
		right&\def\lst@PlaceNumber{%
			\rlap{%
				\normalfont
				\kern\linewidth \kern\lst@numbersep
				\lst@numberstyle{\lst@numberorsymbol}}}%
	}{%
		\PackageError{Listings}{Numbers #1 unknown}\@ehc
	}%
}
\makeatother

% http://www.bollchen.de/blog/2011/04/good-looking-line-breaks-with-the-listings-package/
\newcommand{\shellprebreak}{%
	\raisebox{0ex}[0ex][0ex]{%
		\ensuremath{\hookleftarrow}}}
\newcommand{\shellpostbreak}{%
	\raisebox{0ex}[0ex][0ex]{%
		\ensuremath{\hookrightarrow}\hspace{2pt}}}
\newcommand{\srcprebreak}{%
	\shellprebreak}
\newcommand{\srcpostbreak}{%
	\shellpostbreak}

% stev: \shelllinewidth below should be defined such that all
% shell listings of type '$ hash-trie-test ...' to fit entirely
% on one line; of particular interest is the one obtained from
% \shellinputlisting{hash-trie-test-series-code} -- which is
% the longest one
\newlength\shelllinewidth
\setlength\shelllinewidth{1.35\linewidth}

\newcommand{\shellskipamount}{%
	\baselineskip}
\newcommand{\shellinnerskipamount}{%
	2\parskip}
\newcommand{\shellafterskipamount}{%
	\parskip}
\newcommand{\shellafterskip}{%
	\addvspace{-\shellafterskipamount}}%
\newcommand{\srcskipamount}{%
	\shellskipamount}
\newcommand{\srcafterskipamount}{%
	\shellafterskipamount}
\newcommand{\srcafterskip}{%
	\addvspace{-\srcafterskipamount}}%

\lstdefinestyle{shelllistingstyle}{%
	numbers=none,
	%!!! xleftmargin=3em,
	showstringspaces=false,
	frame=leftline,framesep=1em,framerule=.05pt,%!!!SHELLVERTLINE .1pt,
	basicstyle=\smaller\ttfamily\bfseries,
	breakindent=0pt,breakautoindent=false,
	breaklines=true,breakatwhitespace=true,
	abovecaptionskip=0pt,
	belowcaptionskip=0pt,
	aboveskip=\shellskipamount,
	belowskip=\shellskipamount,
	postbreak=\shellpostbreak,
}
\lstdefinestyle{srclistingstyle}{%
	numbers=left,
	% stev: note that line numbers in pascal listings must
	% align perfectly with the dots of shell listings!
	numbersep=1.3em,
	numberstyle=\tiny,
	showstringspaces=false,
	basicstyle=\smaller\ttfamily,
	breakindent=0pt,breakautoindent=false,
	breaklines=true,breakatwhitespace=true,
	linewidth=\shelllinewidth,lineskip=1.05pt,
	abovecaptionskip=0pt,
	belowcaptionskip=0pt,
	aboveskip=\srcskipamount,
	belowskip=\srcskipamount,
	postbreak=\srcpostbreak,
	prebreak=\srcprebreak,
	emphstyle=\textbf,
}
\newcommand{\msg}[1]{%
	\vspace{-\parskip}%
	\begin{center}%
	{\tt``}\lstinline[basicstyle=\ttfamily]{#1}{\tt''}
	\end{center}%
	\vspace{-\parskip}}
\newcommand{\msgdot}[1]{%
	\msg{#1.}}
\newcommand{\regex}[1]{%
	\textbf/\lstinline[basicstyle=\ttfamily\bfseries]{#1}\textbf/}
\lstnewenvironment{shelllisting}{%
	\lstset{%
		style=shelllistingstyle,
		postbreak=\shellpostbreak,
		prebreak=\shellprebreak,
		linewidth=\shelllinewidth,
		lineskip=1.05pt,
	}%
}{%
	\shellafterskip
}
\newcommand{\srcinline}[1]{%
	\lstinline[
		style=srclistingstyle,
		basicstyle=\ttfamily,
		mathescape=true,
	]{#1}%
}
\lstdefinestyle{paslistingstyle}{%
	style=srclistingstyle,
	showstringspaces=true,
	language=pascal,
	emph={
		define,
		return,
		incr,
		eoln,
		eof
	},
}

\makeatletter
\lstnewenvironment{paslisting}[1][]{%
	\lstset{%
		#1,
		firstnumber=auto,
		style=paslistingstyle,
		escapechar=@,
	}%
    \csname\@lst @SetFirstNumber\endcsname
}{%
	\csname\@lst @SaveFirstNumber\endcsname
	\srcafterskip
}
\makeatother

\lstdefinestyle{cpplistingstyle}{%
	style=srclistingstyle,
	language=[GNU]C++,
	tabsize=4,
	emph={
		alignas,
		alignof,
		auto,
		decltype,
		nullptr,
		constexpr,
		final,
		override,
		static_assert,
		noexcept,
		__func__,
		__FILE__,
		__LINE__,
		__VA_ARGS__,
	},
}
\lstnewenvironment{cpplisting}{%
	\lstset{%
		style=cpplistingstyle,
	}%
}{%
	\srcafterskip
}
\newcommand{\cppinline}[1]{%
	{\lstinline[
		style=cpplistingstyle,
		basicstyle=\ttfamily,
		postbreak=,
		prebreak=,
	]{#1}}%
}

\lstnewenvironment{hashtrielisting}[1][1]{%
	\lstset{%
		literate={!!!INT}{$\ INT\!\!:$}4,
		style=cpplistingstyle,
		showskiplines=true,
		lineskip=-.82pt,
		firstnumber=#1,
		escapechar=@,
	}%
}{%
	\srcafterskip
}
\newenvironment{shelllistinggroup}{%
	\bgroup
	\renewcommand{\shellafterskip}{%
		\addvspace{-\shellinnerskipamount}}%
}{%
	\egroup
	\vspace{\shellinnerskipamount}%
	\shellafterskip
}
\newcommand{\shellinputlisting}[1]{%
	\begin{spacing}{1.05}%
	\lstinputlisting[
		style=shelllistingstyle,
		postbreak=\shellpostbreak,
		prebreak=\shellprebreak,
		linewidth=\shelllinewidth,
	]{#1.txt}%
	\shellafterskip
	\end{spacing}%
}

\newlength\tablerowsep
\setlength\tablerowsep{2pt}

\newlength\tableoutersep
\setlength\tableoutersep{0pt}

\newlength\tablecellsep
\setlength\tablecellsep{2.5pt}

\setlength\cellspacetoplimit{\tablecellsep}
\setlength\cellspacebottomlimit{\tablecellsep}

\setlength\abovetopsep{\tableoutersep}
\setlength\aboverulesep{\tablerowsep}
\setlength\belowrulesep{\tablerowsep}
\setlength\belowbottomsep{\tableoutersep}

\newenvironment{ctabular}[2][\relax]{%
	\begin{center}%
	\ifthenelse{\not\equal{#1}{\relax}}{%
		\label{#1}%
	}{}%
	\newcommand{\head}[1]{\textbf{##1}}%
	\begin{tabular}{#2}%
}{%
	\end{tabular}%
	\end{center}%
}
\newenvironment{ctabularx}[4][\relax]{%
	\center
	\ifdefined\lnwidth\else
	\newlength\lnwidth
	\fi
	\setlength\lnwidth{#2-2\tabcolsep}%
	\newcommand{\multi}[2]{%
		\addlinespace[.5ex]%
		\multicolumn{##1}{p{\lnwidth}}{##2}%
	}%
	\newcommand{\dashrule}{%
		\multicolumn{#4}{@{}l@{}}{%
			\makebox[#2]{%
				\hdashrule[.5ex]{#2}{\lightrulewidth}{1pt}%
				\hfill%
			}%
		}\\%
	}%
	\newcommand{\head}[1]{\textbf{##1}}%
	\ifthenelse{\not\equal{#1}{\relax}}{%
		\label{#1}%
	}{}%
	\tabularx{#2}{#3}%
}{%
	\endtabularx
	\endcenter
}

\newcolumntype{t}{>{\hsize=.35\hsize}X}

\newcommand{\codeelem}[1]{#1}
\newcommand{\define}{\code{{\#}define}}%
\newcommand{\func}{\codeelem{function}}%
\newcommand{\mainfunc}[1]{\codeelem{the \code{main} function of \code{#1}}}%
\newcommand{\macro}{\codeelem{preprocessor macro}}%
\newcommand{\varmacro}{\codeelem{variadic} \macro}%
\newcommand{\varfunc}{\codeelem{variadic function}}%
\newcommand{\printffunc}{\codeelem{\code{printf}-like} \varfunc}%
\newcommand{\printffunctempl}{\printffunc\ \codeelem{template}}%
\newcommand{\functempl}{\codeelem{function template}}%
\newcommand{\cexprfunctempl}{\code{constexpr} \functempl}%
\newcommand{\alias}{\codeelem{type alias}}%
\newcommand{\enum}{\code{enum}}%
\newcommand{\class}{\code{class}}%
\newcommand{\classtempl}{\class\ \codeelem{template}}%
\newcommand{\struct}{\code{struct}}%
\newcommand{\structtempl}{\struct\ \codeelem{template}}%

% stev: tabularx needed \NewEnviron from package environ.sty
% http://tex.stackexchange.com/questions/99612/using-tabularx-in-a-custom-environment-misplaced-noalign/111488#111488
% Placing any alignment material (including \hline) in the end
% clause here doesn't work, I think because of the way tabularx
% scans ahead for the table material. [by Andrew Swann]
\NewEnviron{ntabular}{%
	\bgroup
	\newcommand{\erow}{\\\dashrule}%
	\newcommand{\desc}[1]{\\\multi{3}{##1}}%
	\renewcommand{\codeelem}[1]{\emph{##1}}%
	\renewcommand{\cexprfunctempl}{\cppinline{constexpr} \functempl}%
	\renewcommand{\enum}{\cppinline{enum}}%
	\renewcommand{\class}{\cppinline{class}}%
	\renewcommand{\struct}{\cppinline{struct}}%
	\begin{ctabularx}{.95\textwidth}{StSlX}{3}%
	\toprule
	\head{name \& description} & \head{type}\\
	\midrule
	\addlinespace[1ex]%
	\BODY\\
	\addlinespace[1ex]%
	\bottomrule
	\end{ctabularx}%
	\egroup
}
\NewEnviron{dtabular}{%
	\bgroup
	\newcommand{\erow}{\\\dashrule}%
	\newcommand{\pgparalno}[3]{\pg{##1}, \para{##2}, \lno{##3}}%
	\newcommand{\pgreflno}[1]{p.~\pageref{##1}, \lno{\ref{##1}}}%
	\newcommand{\desc}[1]{\\\multi{4}{##1}}%
	\begin{ctabularx}{.95\textwidth}{StSlSlX}{4}%
	\toprule
	\head{name \& description} & \head{\pascal\ source} & \head{\cplusplus\ source}\\
	\midrule
	\addlinespace[1ex]%
	\BODY\\
	\addlinespace[1ex]%
	\bottomrule
	\end{ctabularx}%
	\egroup
}

% customize varioref's \vpageref and \vpagerefrange commands:
\def\reftextfaceafter {the \reftextvario{facing}{next} page}
\def\reftextfacebefore{the \reftextvario{facing}{preceding} page}
\def\reftextafter     {the \reftextvario{following}{next} page}
\def\reftextbefore    {the \reftextvario{preceding}{previous} page}
\def\reftextcurrent   {\reftextvario{this}{the current} page}
\def\reftextfaraway#1{page \pageref{#1}}
\def\reftextpagerange#1#2{pages \pageref{#1}\~\pageref{#2}}
\def\reftextlabelrange#1#2{\ref{#1} to \ref{#2}}

\def\makepagerefrangetrue{%
	\let\ifpagerefrange\iftrue}
\ifdefined\ifpagerefrange\else
\makepagerefrangetrue
\fi
\ifpagerefrange
\newcommand{\pagerefrange}[2]{%
	\vpagerefrange{#1}{#2}}
\else
\newcommand{\pagerefrange}[2]{}
\fi

\quotingsetup{
	%font=smaller,
	vskip=0pt,
}

\renewcommand{\contentsname}{Table of Contents}
\renewcommand{\cftsecleader}{\cftdotfill{\cftdotsep}}

\newlength\citespace
\setlength\citespace{.3pt}
\renewcommand{\citeleft}{\mbox{\textbf{[}\hspace{\citespace}}}
\renewcommand{\citeright}{\mbox{\hspace{\citespace}\textbf{]}}}
\renewcommand{\citeform}[1]{\textbf{#1}}

\makeatletter
\renewcommand\@biblabel[1]{\textbf{[#1]}}%
\makeatother
\newenvironment{bbibliography}[1]{%
	\begin{thebibliography}{\textbf{#1}}%
	\DeclareUrlCommand\url{\urlstyle{tt}\vspace{\parskip}\\}
}{%
	\end{thebibliography}%
}

% http://tex.stackexchange.com/questions/29261/how-to-suppress-typesetting-of-section-title/29266#29266
\newcommand{\nosection}[1]{%
	\refstepcounter{section}%
	\addcontentsline{toc}{section}{\protect\numberline{\thesection}#1}%
	\markright{#1}%
}

\title{The \Hashtrie\ of Knuth \& Liang:\\A \cplusplus11 Implementation}
\author{\larger\vargyas\medskip\\ \tt\textbf{stvar@yahoo.com}}
\date{Oct 31, 2015}

\setstretch{.9}
\setlength{\footnotemargin}{1.1em}

% http://tex.stackexchange.com/questions/236781/align-the-footnote-markers-to-the-right-in-footnotes/236802#236802
\makeatletter
\patchcmd{\@makefntext}
	{\@makefnmark\hss}
	{\hss\@makefnmark\hspace{4pt}}
	{}{}
\makeatletter

\begin{document}
\maketitle
\tableofcontents
%
\section{The \HashTrie\ Program}
\label{hash-trie-program}
%
The \HashTrie\ program's vocation is that of a \cplusplus\ tool for exercising
the interesting \emph{hash\=trie} data structure of Knuth \cite[\pg{157}]{knuth:literate}
-- a variant of the so called dynamic packed trie data structure of
Liang \cite[\pg{32}]{liang:hyphenation}.
\HashTrie\ shall provide an \emph{implementation} for this data structure in modern
\cplusplus\ -- that is the new \cplusplus11 language \cite{stroustrup:tc++pl4} --
and shall encompass the implementation into a framework serving the function of
\emph{test bed} for various easily definable \emph{configurations} of hash-tries.

For the purpose of having an \emph{implementation} of the \hashtrie*
data structure which facilitates the ease of definition of different
\emph{configurations} of the structure, upon a few initial iterations over
Knuth's material \cite[\pp{151}{167}]{knuth:literate}, I designed
and implemented a set of ``modules'', of which the leading one -- the
\code{HashTrie} module --, being parametrized statically by a handful
of \emph{configuration parameters}, gives the means for the birth of
as many concrete \hashtrie*\ structures as needed. For the sake
of simplicity all modules were hosted by a single source file,
\code{hash-trie.cpp}.

As a result of compiling the sole source file \code{hash-trie.cpp}
with configuration parameters set according to the needed configuration
of \hashtrie*, a binary \prog{hash-trie} comes to existence, which will
further be used by the \emph{test bed} for testing purposes.

One important driving force behind designing the modules of \code{hash-trie.cpp}
the way they were, was the desire to have a clear-cut separation between the
major parts of the program:
%
\begin{ctabular}[initial-design-modules]{ScSl}
\toprule%!!!\cmidrule[\heavyrulewidth](r){1-1}\cmidrule[\heavyrulewidth](l){2-2}
\head{module}       & \head{contents}\\
\midrule%!!!\cmidrule(r){1-1}\cmidrule(l){2-2}
\textbf{Sys}        & utilities close to the undelying system libraries\\
\textbf{Ext}        & extensions of the standard \cplusplus\ library\\
\textbf{\HashTrie*} & everything belonging to the \hashtrie*\ data
                      structure\\
\textbf{Main}       & the \code{main} function of \prog{hash-trie} along with
                      a few related utilities\\
\bottomrule%!!!\cmidrule[\heavyrulewidth](r){1-1}\cmidrule[\heavyrulewidth](l){2-2}
\end{ctabular}
%
\section{The \Cplusplus11 Implementation of \HashTrie}
\label{hash-trie-c++11-impl}
%
The first objective -- set forth at the beginning of section
\ref{hash-trie-program}, \nameref{hash-trie-program} -- is that of providing an
\cplusplus11 implementation for the \hashtrie*\ data structure of Knuth~\&~Liang.
This data structure in defined and implemented in chapter 6, \emph{Programming Pearls,
Continued: Common Words (1986)}, of \cite[\pg{151}]{knuth:literate} as core part
of a program solving the following problem proposed by Bentley \cite{bentley:pearls}:
%
\begin{quoting}
Given a text file and an integer \emph{k}, print the \emph{k} most common words
in the file (and the number of their occurrences) in decreasing frequency.
\end{quoting}
%
Knuth states in \parapag[Dictionary Lookup]{17}{157},
of his work, that the \hashtrie*\ structure is a variant of a data
structure due to Liang's Ph.D. work -- see section \emph{Dynamic packed tries}
of chapter \emph{Pattern generation} in \cite[\pg{32}]{liang:hyphenation}.
%
\subsection{The \pascal\ Implementation of Knuth}
\label{knuth-pascal-impl}
The general structure of Knuth's \pascal\ program\footnote{To be more precise,
Knuth implemented his solution to Bentley's problem in his own \code{Web} system
of programming (see chapter 4, \emph{Literate Programming}, of \cite{knuth:literate}
for a thorough description of this system). \code{Web} boils down the source
\emph{literate program} given as input to a completely defined \pascal\ source program.}
is revealed in \parapag{3}{152}:
%
\begin{paslisting}
program common_words(input, output);
type { Type declarations (17) }
var { Global variables (4) }
{ Procedures for initialization (5) }
{ Procedures for input and output (9) }
{ Procedures for data manipulation (20) }
begin
{ The main program (8) }
end.
\end{paslisting}
%
Upon looking deeper into the implementation, one can see that,
the \emph{global variable}s constitute the main mechanism for communication
between the parts -- \ie\ the \code{procedure}s and the \code{function}s -- of
the program, or even, within one part of it only. This is to say that the program does
not encapsulate related data into, \eg, \code{record} structures -- thus, nor
does he attach specific \code{procedure}s and \code{function}s to such structures --,
for to have a more clear-cut division between the relevant pieces of functionality
of the program.

Another trait of this \pascal\ program is that system dependecies -- as these
are implied by the potential necessity of running the program on several different
supporting platforms -- are not dealt with at all. This is of course justifiable
by the fact that the \pascal\ language and it's environment -- at the time of
Knuth's writing -- had no concern at all with these kind of issues.\footnote{For
an interesting and detailed yet debatable analysis of \pascal's shortcomings
look upon the well-known report of Kernighan \cite{kernighan:pascal}. Note that
the paper appeared before ISO/IEC 7185 -- the \pascal\ standard -- was created
in 1983.} Nevertheless, Knuth is well aware of the matters of portability and,
nota bene, he does have his own solutions to such kind of difficulties in the
case of the complex \textbf{\TeX} system of typesetting he has been built
\cite[ch.~4, sec.~\emph{Portability}, \pp{122}{124}]{knuth:literate}.

The third trait of this \pascal\ program is concerning the issue of handling and
reporting the error conditions of at least two kinds: logic errors and \runtime\
errors. (Simply put, the category of logic errors refers to those errors which
are due to programming mistakes within the internal logic of the program; the
category of \runtime\ errors encompasses the kind of fault conditions due to
the outer world of the program -- these are either conditions originating
from the supporting environment or from the user mishandling the program.)
With respect to the category of logic errors, the program relies completely on
\pascal's type system: there are no supplemental checks of this kind made
in the code.
%
%\pag{101}:
%\begin{quoting}
%[...] Furthermore \code{WEB} itself has a macro-processing ability that
% makes \Pascal's limitations largely irrelevant.
%\end{quoting}
%
%\pag{121}
%\begin{quoting}
%[...] This feature of \code{WEB} was introduced specifically to overcome
%\pascal's unfortunate inability to do compile-time arithmetic.
%\end{quoting}
%
%\parapag{6}{153}
%\begin{quoting}
%\textbf{6.} The \code{WEB} system, which may be thought of as a preprocessor
%for \Pascal, includes a macro definition facility so that portable programs
%are easier to write. 
%\end{quoting}
%

On the other hand, concerning the \runtime\ errors, the code handles
very scarcely the errors which are due to the user misusing the program.
For example, look at the code implementing the function \code{read\_int}
in \parapag{9}{154}: there is an insufficient wary about the user providing
an invalid numerical input, and there is none with regards to an otherwise
valid numerical input that could overflow the capacity of the internal
\code{integer} type involved: 
%
\begin{paslisting}
{ 9. Basic input routines. [...] a function
  that reads an optional positive integer,
  returning zero if none is present at the
  beginning of the current line. }
function read_int: integer;
var n: integer; { the accumulated value }
begin
  n := 0;
  if not eof then
  begin
    while (not eoln) and (input^ = ' ') do
      get(input);
    while (input^ >= '0') and (input^ <= '9') do
    begin
      n := 10 * n + ord(input^) - ord('0');
      get(input);
    end;
  end;
  read_int := n;
end;
\end{paslisting}
%
In this context, it is to be noted the interesting remark Knuth made\footnote{%
Knuth's remark echoes Kernighan's own accounts about \pascal's environment, and,
particularly, about \pascal's \code{I/O} subsystem. (see section 4 of Kernighan's
report.)} at the beginning of the section which includes the code above
(\parapg{9}{154}):
%
\begin{quoting}
It will be nice to get the messy details of \pascal\ input out of the way
and off our minds.
\end{quoting}
%
The other input procedure of the program is \code{get\_word} (see it
below, on \vpageref{get-word-func}). Similarly to the function \code{read\_int}
above, the code of \code{get\_word} pays little concern to the \runtime\ errors
that might occur due to the faulty input given.

Having in mind the remarks made above, for refining farther the first of the
two main objectives stated in section \ref{hash-trie-program}, I choose for
the \cplusplus11 implementation of \hashtrie*\ to be closely attached
to Knuth's own \pascal\ implementation, but \emph{no closer}.

The point here is that my intention is not to make a \onetoone\ translation of
\pascal\ code to \cplusplus, but a translation that is made up of choices resulted
upon careful deliberation of which \pascal\ construct is transformed to what
\cplusplus\ equivalent construct. An illustrating example might be the following:
the implementation in \parapag{13}{156} defines several things as shown by
the code below:
%
\begin{paslisting}
define max_word_length = 60 { words shouldn't be longer than this }
buffer: array [1..max_word_length] of 1..26; { the current word }
word_length: 0..max_word_length;
       { the number of active letters currently in buffer }
word_truncated: boolean;
       { was some word longer than max_word_length? }
\end{paslisting}
%
% % stev: the alignment of line numbers in pascal listings
% % must match precisely that of the dots in shell listings
% \begin{shelllisting}
% $ foo
% bar
% $
% \end{shelllisting}
%
where the nearly surrounding context is as follows:
%
\begin{quoting}
\textbf{13.}
Each new word found in the input will be placed into a \code{buffer} array.
We shall assume that no words are more than 60 letters long; if a longer word
appear, it will be truncated to 60 characters, and a warning message will be
printed at the end of the run.
\end{quoting}
%
From my perspective, it is obvious that the \cplusplus\ implementation shouldn't
impose such an uptight limitation on the length of the input words. On the
other hand, yet more importantly, is that the implementation shoudn't define
\code{buffer} to be of type \code{char[max\_word\_length+1]}, but to be 
\code{char*} -- \ie\ a null\=terminated string -- and/or \code{std::string}.
Such an option of transformation from \pascal\ to \cplusplus\ would render
\code{max\_word\_length},
\code{word\_length} and
\code{word\_truncated}
as useless and, consequently, they will not be translated into the implementing
\cplusplus\ code.
Moreover, deriving from this choice of translation, the \code{I/O} \pascal\ code
of \paraspages{15}{16}{156}{157} which is listed below:
%
\begin{paslisting}[label=get-word-func]
{ 15. We're now ready for the main input routine,
  which puts the next word into the buffer. If no more
  words remain, word_length is set to zero; otherwise
  word_length is set to the length of the new word. }
procedure get_word;
label exit; { enable a quick return }
begin
  word_length := 0;
  if not eof then
  begin
    while lettercode[ord(input^)] = 0 do
      if not eoln then
        get(input)
      else
      begin
         read_ln;
         if eof then return;
      end;
    { Read a word into buffer (16) }
    { 16. At this point lettercode[ord(input^)] > 0,
      hence input^ containts the first letter of a word. }
    repeat
      if word_length = max_word_length then
        word_truncated := true
      else
      begin
        incr(word_length);
        buffer[word_length] := lettercode[ord(input^)];
      end;
      get(input);
    until lettercode[ord(input^)] = 0;
  end;
exit:
end;
\end{paslisting}
%
will be translated to something simply like the \cplusplus\=styled \code{I/O}
statements:
%
\begin{cpplisting}
std::string  input_word;
std::istream input_stream = ...;
std::getline(input_stream, input_word);
\end{cpplisting}
%
The string \code{input\_word} will than be \emph{passed by value} to where is
needed. Furthermore, is it easily foreseeable that the mapping provided by the
\code{lettercode} table (\paraspages{11}{12}{155}{156}, see below) should
be deferred to a \emph{traits} class -- which will become a default template
parameter of and it will be used by the class template that will encompass the
core functionality of the \hashtrie*\ data structure.

One more remark concerning the peculiarities of \pascal\ is referring to, in
Knuth's words, the following things (\paraspag{11}{12}{155}):
%
\begin{quoting}
\textbf{11.} To find words in the \emph{input} file, we want a quick way to
distinguish letters from nonletters. \pascal\ has conspired to make this problem
somewhat tricky, because it leaves many details of the character set undefined.
[...]

If \code{c} is a value of type \code{char} that represents the \code{k}th
letter of the alphabet, then \code*{lettercode[ord(c)]\,=\,k}; but
if \code{c} is a nonletter, \code*{lettercode[ord(c)]\,=\,0}. We assume
\code*{0\,$\le$\,ord(c)\,$\le\,$255} whenever \code{c} is of type \code{char}.
\end{quoting}
%
\vspace{-\partopsep}
%
\label{pascal-lettercode-begin}
\begin{paslisting}[name=lettercode]
{ Global variables (4) += }
lowercase, uppercase: array [1..26] of char; { the letters }
lettercode: array [0..255] of 0..26; { the input conversion table }
\end{paslisting}
%
\vspace{-\partopsep}
%
\begin{quoting}
\textbf{12.} A somewhat tedious set of assignments is necessary for the
definition of \code{lowercase} and \code{uppercase}, because letters need not
be consecutive in \pascal's character set.
\end{quoting}
%
\vspace{-\partopsep}
%
\begin{paslisting}[name=lettercode]
{ Set initial values (12) }
lowercase[1] := 'a'; uppercase[1] := 'A';
...
lowercase[26] := 'z'; uppercase[26] := 'Z';
for i := 0 to 255 do
  lettercode[i] := 0;
for i := 1 to 26 do
begin
  lettercode[ord(lowercase[i])] := i;
  lettercode[ord(uppercase[i])] := i;
end;
\end{paslisting}
\label{pascal-lettercode-end}
%
Indeed, \code{lowercase} and \code{uppercase} are not necessary in the
environment of \cplusplus: the implementation will simply use either
\code{tolower} or \code{towlower} functions. Similarly to the case of
\code{lettercode}, the mapping provided by \code{lowercase} will be deferred
to the traits class associated to the \hashtrie*\ structure.

For the category of \onetoone\ translations from \pascal\ to \cplusplus,
hereafter will follow a couple of examples of places where the \pascal\
implementation matches quite well the translated \cplusplus\ side. As a matter
of fact these places are part of the core algorithms and inner data structures
of \hashtrie*\ -- for which I decided that the \cplusplus\ side be as
close as possible to the \pascal\ side as it was laid down by Knuth.

\pascal's type system is known
to be of a \emph{strong} kind. Unfortunately, at times this type system seems
to be \emph{too strong}: \eg\ the size of an array or string is part of its type,
thus making certain programming tasks quite difficult.\footnote{Subsection 2.1
of \cite{kernighan:pascal} accounts for this problem out of the experience
the author had rewriting a weighty set of programs in \pascal.}
Above was shown that, for certain cases, \code{array}'s
of fixed size in \pascal\ where not translated \onetoone\ to \cplusplus, but
an adaptation has been made such that the translation to suite the target style
of programming and the surrounding environment.

However, this is not to be generalized completely, because, for a couple of
inner structures of \hashtrie*\ -- indeed, \code{array}'s of fixed size --, the
\cplusplus\ implementation will keep its declarations tightly parallel with the
originating \pascal\ ones. These declarations are indicated below, extracted
from \parapag[Dictionary Lookup]{17}{158}:
%
\label{pascal-types-begin}
\begin{paslisting}[name=pascal-types]
define trie_size = 32767 { the largest pointer value }@\label{pas-types-first-line}@
pointer = 0..trie_size;
\end{paslisting}
%
from \parapag{18}{159}:
%
\begin{paslisting}[name=pascal-types]
define empty_slot = 0
define header = 27
link, sibling: array [pointer] of pointer;
ch: array [pointer] of empty_slot..header;
\end{paslisting}
%
from \parapag{24}{160}:
%
\begin{paslisting}[name=pascal-types]
define tolerance = 1000
\end{paslisting}
%
and from \parapag[The frequency counts]{32}{162}:
%
\begin{paslisting}[name=pascal-types]
define max_count = 32767 { count's won't go higher then this }
count: array [pointer] of 0..max_count;@\label{pas-types-last-line}@
\end{paslisting}
\label{pascal-types-end}
%
Two of the \code{define}s above -- \code{trie\_size} and \code{tolerance} --
will become part of the \emph{configuration parameters} of \hashtrie*'s generic
structure: out of specification of each of them, concrete \hashtrie*s will
be obtained.

The next two sections to come will introduce a few more of these configuration
parameters. Anticipating a bit, but in direct relation with the \pascal\ context
of this section, one should observe the necessity of at least two more
parameters -- let's name them \code{ARRAY\_BOUNDS} and \code{STRICT\_TYPES}.

These parameters associate with two \pascal\ matters which have to be dealt with:
arrays indexing and arithmetical expressions. The \pascal\ \runtime\ support
ensures that arrays indexing are within the bounds defined for the type involved
and that arithmetical expressions on \code{integer}s do no exceed the limits of
the respective types. The \cplusplus\ code should be flexible enough implementing
transparently these kinds of \runtime\ checks, yet, it should be able to let these
supplemental checks out of a particular build of \hashtrie*\ when efficiency is
of paramount importance.
%
\subsection{The Structure of \code{hash-trie.cpp} Source File}
\label{struct-hast-trie-source}
%
The \cplusplus\ source file of \HashTrie\ was structured according to the initial
design (see the table on \vpageref{initial-design-modules}). Each of the ``modules''
\module{Sys}, \module{Ext} and \module{HashTrie} were encompassed within a namespace
with the same name. However, the handful members of \module{Main} module were
defined at the global scope:
%
\begin{ntabular}
%
\code{program} and \code{verdate}
                          & \cppinline{const char[]}
\desc{The sole global constants of the program. Define the name and
the version number and date of the program. The constant \code{program}
is defined by the build parameter \code{PROGRAM} -- which is controlled
by \prog{make}.}
\erow

\code{global\_options\_t} & \struct
\desc{Encapsulate all global option variables in one structure.}
\erow

\code{globals}            & \code{global\_options\_t}
\desc{The sole global variable. Give access to the global options.}
\erow

\code{options\_t}         & \class
\desc{Parse command line options, update \code{globals} and
provide the action option. These things are obtained upon invoking
the sole publicly defined method of the class:
\centerline{\mbox{\cppinline{static const options_t options(int argc, char* argv[])}.}}}
\erow

\code{print\_config}      & \func
\desc{Print out the configuration parameters of the program.}
\erow

\code{print\_op\_types<>} & \functempl
\desc{Print out the types corresponding to the named operation.}
\erow

\code{print\_types}       & \func
\desc{Print out the types of \emph{add} and \emph{sub} operations by
calling \code{print\_op\_types<>}.}
\erow

\code{hash\_trie\_error}  & \printffunc
\desc{Print out formatted error messages and, if told to do so, exit
the program.}
\erow

\code{hash\_trie\_t}      & \alias
\desc{\code{HashTrie::HashTrie<Sys::char\_t>}.}
\erow

\code{print\_func\_t}     & \alias
\desc{\code{void (hash\_trie\_t::*)(Sys::ostream\&) const}.}
\erow

\code{exec\_hash\_trie}   & \func
\desc{Load up words read from \code{stdin} in an instance of
\code{hash\_trie\_t} and than call on that instance for the named
print function, if given. When the \code{STATISTICS} configuration
parameter was enabled at compile-time, if \code{globals.print\_stats}
says so, then also print out the statistics information of the instance.}
\erow

\code{main}               & \mainfunc{hash-trie.cpp}
\desc{Dispatch the action option obtained from function
\code{options\_t::options} by calling the appropriate
handler function:
\code{print\_config},
\code{print\_types} or
\code{exec\_hash\_trie}.}
%
\end{ntabular}
%
The complete definition of \code{main} function is as follows:
%
% $ sha1sum hash-trie.cpp
% 1901535f6047804749deb5bbf5beb13845b486b2  hash-trie.cpp
% $ hash-trie-source --main-func --tex -t
%
\begin{hashtrielisting}[3429]
int main(int argc, char* argv[])
{
	const auto opt =
		options_t::options(argc, argv);

	switch (opt.action) {
	case options_t::print_config:
		print_config();
		break;
	case options_t::print_types:
		print_types();
		break;
	case options_t::load_trie_only:
		exec_hash_trie();
		break;
	case options_t::print_trie:
		exec_hash_trie(
			&hash_trie_t::print);
		break;
	case options_t::dump_trie:
		exec_hash_trie(
			&hash_trie_t::dump);
		break;
	default:
		SYS_UNEXPECT_ERR(
			"action='%zu'",
			Ext::size_cast(opt.action));
	}

	return 0;
}
\end{hashtrielisting}
%
The action options relating to \code{hash\_trie\_t} are dispatched 
to the function \code{exec\_hash\_trie}:
%
% $ hash-trie-source --exec-hash-trie --tex -t
%
\begin{hashtrielisting}[3394]
using hash_trie_t = HashTrie::HashTrie<Sys::char_t>;

using print_func_t = void (hash_trie_t::*)(Sys::ostream&) const;

void exec_hash_trie(print_func_t print_func = nullptr)
{
	size_t lno = 1;
	hash_trie_t trie;
	hash_trie_t::string_t str;
	while (Ext::getline(Sys::cin, str)) {
		try {
			if (!trie.put(str) && !globals.quiet)
				hash_trie_error(
					false, lno,
					"failed to put '%"
					SYS_CHAR_TYPE_FMTS
					"s' in trie",
					str.c_str());
		}
		catch (const HashTrie::Error& err) {
			hash_trie_error(
				true, lno, "%s", err.what());
		}
		lno ++;
	}

	if (print_func)
		(trie.*print_func)(Sys::cout);

#ifdef CONFIG_HASH_TRIE_STATISTICS
	if (globals.print_stats)
		trie.print_stats(Sys::cout);
#endif
}
\end{hashtrielisting}
%
The namespace \code{Sys} contains system-related declarations and definitions:
%
\begin{ntabular}
%
\code{SYS\_CHAR\_TYPE\_NAME} & \define
\desc{Specify the type used for character representation throughout the program.
Depending on the value of \code{CONFIG\_HASH\_TRIE\_CHAR\_TYPE} being
\code{0} or \code{1}, is either \code{char} or \code{wchar\_t}.}
\erow

\code{SYS\_CHAR\_TYPE\_FMTS} & \define
\desc{Specify the length modifier of a \code{`\%s'} format specifier.
Depending on the value of \code{CONFIG\_HASH\_TRIE\_CHAR\_TYPE} being
\code{0} or \code{1}, is either the empty string or \code{"l"}.}
\erow

\code{Sys::char\_t} & \alias
\desc{Define the type used for character representation:
\code{SYS\_CHAR\_TYPE\_NAME}.}
\erow

\code{Sys::ostream} & \alias
\desc{Define the I/O streams type used for output:
\code{std::basic\_ostream<Sys::char\_t>}.}
\erow

\code{Sys::istream} & \alias
\desc{Define the I/O streams type used for input:
\code{std::basic\_istream<Sys::char\_t>}.}
\erow

\code{Sys::cin} & \code{Sys::istream\&}
\desc{Define the I/O streams object associated with \code{stdin}.
Depending on the value of \code{CONFIG\_HASH\_TRIE\_CHAR\_TYPE},
it refers to \code{std::cin} or to \code{std::wcin}.}
\erow

\code{Sys::cout} & \code{Sys::ostream\&}
\desc{Define the I/O streams object associated with \code{stdout}.
Depending on the value of \code{CONFIG\_HASH\_TRIE\_CHAR\_TYPE},
it refers to \code{std::cout} or to \code{std::wcout}.}
\erow

\code{Sys::cerr} & \code{Sys::ostream\&}
\desc{Define the I/O streams object associated with \code{stderr}.
Depending on the value of \code{CONFIG\_HASH\_TRIE\_CHAR\_TYPE},
it refers to \code{std::cerr} or to \code{std::wcerr}.}
\erow

\code{Sys::die} & \printffunc
\desc{Handle fatal error conditions: print out a formatted
error message to \code{stderr} and than exit the program.
If \code{globals.dump\_backtrace} says so, print out
the backtrace of the program prior to bailing out.}
\erow

\code{Sys::assert\_failed} & \func
\desc{Handle an assertion failed error condition: invoke \code{die}
with a proper argument list.}
\erow

\code{Sys::unexpect\_error} & \printffunc
\desc{Handle an unexpected error condition: invoke \code{die}
with a proper argument list.}
\erow

\code{SYS\_ASSERT} & \macro
\desc{Test for the expression provided as argument to be \code{true}:
if not, raise an assertion failed error by calling in \code{Sys::assert\_failed}.
Nothing is done when \code{DEBUG} is not defined.}
\erow

\code{SYS\_UNEXPECT\_ERR} & \varmacro\footnote{%
Variadic preprocessor macros via \code{\_\_VA\_ARGS\_\_} are by now legal in
\cplusplus11: ISO/IEC 14882:2011, 16.3 Macro replacement, pts.~4, 5 and 10;
16.3.1 Argument substitution, pt.~2; 16.3.5 Scope of macro definitions, pt.~9.}
\desc{Raise an unexpected error by calling in \code{Sys::unexpect\_error}.}
%
\end{ntabular}
%
\newpage
% 
\begin{ntabular}
%
\code{Sys::error<>} & \printffunctempl
\desc{Format an error string and than throw it via an exception object
of type specified.}
\erow

\code{Sys::clocks\_t} & \struct
\desc{Encapsulate three time counters of microsecond resolution:
\code{real}, \code{user} and \code{sys}. Define convenient
operators: \code{+}, \code{+=}, \code{-}, \code{-=} and,
for printing out, \code{<<}.
Is compiled in only when defining \code{CONFIG\_HASH\_TRIE\_STATISTICS}.}
\erow

\code{Sys::clock\_t} & \struct
\desc{For printing it out, restrict to the specified one of the three
time counters of a \code{clocks\_t} object.
Is compiled in only when defining \code{CONFIG\_HASH\_TRIE\_STATISTICS}.}
\erow

\code{Sys::utime\_t} & \class
\desc{Define a convenient interface to the system API \code{gettimeofday}
and \code{getrusage} functions for time measurements of microsecond resolution.
Return a \code{clocks\_t} object.
Is compiled in only when defining \code{CONFIG\_HASH\_TRIE\_STATISTICS}.}
%
\end{ntabular}
%
The namespace \code{Ext} contains many useful extensions -- function templates
and class templates -- to the standard \cplusplus\ library. However, the table
that follows was shortened for to list only the most important of these
extensions:
%
\begin{ntabular}
\code{Ext::min<>},
\code{Ext::max<>} & \cexprfunctempl s
\desc{Provide handy shortcuts to the corresponding \code{std::numeric\_limits<>}
functions.}
\erow

\code{Ext::digits<>} & \cexprfunctempl
\desc{Provide a handy shortcut to the \code{std::numeric\_limits<>::digits}
constant.}
\erow

\code{Ext::size\_cast<>} & \functempl: \code{size\_t(T)}
\desc{Cast safely an integer of type narrower or equal width than
\code{size\_t} to \code{size\_t}.}
\erow

\code{Ext::array<>} & \functempl: \code{Sys::array\_t<T,N>(T(\&)[N])}
\desc{Access \cplain-style arrays via a range checking \code{operator[]}.}
\erow

\code{Ext::array\_size<>} & \cexprfunctempl: \code{size\_t(T(\&)[N])}
\desc{Return the number of elements of \cplain-style arrays.}
\erow

\code{Ext::getline<>} & \functempl
\desc{Replace \code{std::getline<>} on \code{std::basic\_string<>} with
an optimized version.\footnote{New in \cplusplus11 is that \code{std::string} of
\code{char}-like objects is required to store contiguously its elements
(see ISO/IEC 14882:2011, 21.4.1 pt.~5).
A consequence of this is that the growing of a \code{std::basic\_string<>}
inside \code{Ext::getline<>} can be done efficiently in an I/O buffer-like
manner.}}
\erow

\code{Ext::box\_t<>} & \classtempl
\desc{Provide a wrapper class around integer types for to perform bounds
checkings on the assignments and overflow and underflow checkings on the
arithmetic operations involved.}
\end{ntabular}
%
%!!!\newpage
%
The namespace \code{HashTrie} contains everything related to the \hashtrie*
data structure. A short synopsis follows below. The next subsection will
delve deeply into the implementation code.
%
\begin{ntabular}
\code{HashTrie::Error} & \struct 
\desc{Define the exception class used for signaling the few run-time error
conditions that might occur in \code{HashTrie::HashTrie<>} class. It is derived
from \code{std::runtime\_error}.}
\erow

\code{HashTrie::types\_t} & \enum
\desc{Name the four main types used by class \code{HashTrie::HashTrie<>}: 
\code{count\_type},
\code{pointer\_type},
\code{letter\_type} and
\code{cell\_type}.\footnote{The \pascal\ types to which the names defined by
\code{HashTrie::types\_t} are referring to were shown by the listings on
\pagerefrange{pascal-types-begin}{pascal-types-end}.
More details about the \cplusplus\ counterparts of these types will be given
in the next subsection.}}
\erow

\code{HashTrie::types\_traits\_t} & \structtempl\ \code{<typename T, types\_t t>}
\desc{Define the underlying integer type and the associated limits for the named
type \code{t}.}
\erow

\code{HashTrie::add<>} & \functempl
\desc{Define an addition operation, transparently over the boxed or just plain
underlying unsigned integer types used by \code{HashTrie::HashTrie<>}. Implement
the operation such a way that the type of the resulting value be wide enough
to avoid overflowing.}
\erow

\code{HashTrie::sub<>} & \functempl
\desc{Define an subtraction operation, transparently over the boxed or just plain
underlying unsigned integer types used by \code{HashTrie::HashTrie<>}. Implement
the operation such a way that the type of the resulting value be wide enough
to avoid underflowing.}
\erow

\code{HashTrie::char\_traits\_t} & \structtempl\ \code{<typename C = char>}
\desc{Define the default traits class template associated to the character
type used by \code{HashTrie::HashTrie<>}.\footnote{The class template
\code{HashTrie::char\_traits\_t<>}
relates closely to the issues of tables
\code{lowercase},
\code{uppercase} and
\code{lettercode} in the \pascal\ implementation. (see the accounts given
on \pagerefrange{pascal-lettercode-begin}{pascal-lettercode-end}.)}}
\erow

\code{HashTrie::size\_traits\_t} & \struct
\desc{Encompass a few static parameters of class \code{HashTrie::HashTrie<>}:
\code{max\_count},
\code{trie\_size} and
\code{tolerance}. The last two of these are defined by corresponding
\code{CONFIG\_HASH\_TRIE\_*} configuration parameters.}
\erow

\code{HashTrie::HashTrie<>} & \classtempl
\desc{Implement the main functionality of the \emph{hash-trie} data structure
of Knuth \& Liang.}
\end{ntabular}
% 
\subsection{The Implementation of \code{HashTrie<>} Class Template}
\label{hash-trie-class-templ-impl}
%
This section is devoted to plunging into the most concrete details of the
implementation code of the \hashtrie* structure of Knuth \& Liang. The previous
section glimpsed over the members of \code{HashTrie} namespace which accomodate
the class \code{HashTrie<>} and its immediate relatives. Below is listed
the \cplusplus\ code for the definitions of
class \code{HashTrie<>},
of \structtempl\ \code{type\_traits\_t},
of \structtempl\ \code{char\_traits\_t},
and of \struct\ \code{size\_traits\_t}.
Also is listed the implementation code
for the constructor of the \code{HashTrie<>} class,
for its public method \code{put} and
for its most important function, the private method \code{find}.

The listings were shortened a bit by eliminating certain peripheral inline
function code and all static assertions, comment lines and debugging code.\footnote{%
The listing parts of \code{hash-trie.cpp} seen in this section and in the previous
one are annotated with actual line numbers from the source file itself.}

For the implementation code in \code{hash-trie.cpp}, I took into account the
newest corrections of the \pascal\ implementation of \hashtrie*\, as published
by Knuth on his own site.\footnote{%
\url{http://www.cs.stanford.edu/~uno/lp.html}: the web page of \cite{knuth:literate}
-- contains the errata of the latest printing of the book, the sixth, of 2013;\\
\url{http://www.cs.stanford.edu/~uno/lp-err.ps.gz}: the errata for the first
printing of the book.} (see the source lines
\ref{cpp-make-alpha-expr} and \ref{cpp-init-count-zero} below.)

The \pascal\ type and variable definitions on
\pagerefrange{pascal-types-begin}{pascal-types-end},
as was already emphasized, are part of the core structures of \hashtrie*.
These definitions imply four kinds of \emph{integer subrange} types to be
used by the \cplusplus\ implementation: (The subrange type \code{1..26} is
brought forward by the definitions of tables
\code{lowercase},
\code{uppercase} and
\code{lettercode}
on \vpageref{pascal-lettercode-begin}.)
%
\newcommand{\unnamed}{\emph{unnamed}}%
%
\begin{ctabular}{ScScScSc}
\toprule
\head{\pascal\ name} & \head{\pascal\ subrange type} & \head{\cplusplus\ \enum\ name}\\
\midrule
\unnamed             & \code{0..max\_count}          & \code{count\_type}\\
\code{pointer}       & \code{0..trie\_size}          & \code{pointer\_type}\\
\unnamed             & \code{1..26}                  & \code{letter\_type}\\
\unnamed             & \code{empty\_slot..header}    & \code{cell\_type}\\
\bottomrule
\end{ctabular}
%
The \cplusplus\ types which correspond to the \pascal\ subrange types above are
defined through the \structtempl\ \code{types\_traits\_t}. For each \enum\ name
there is a specialized version of the \code{types\_traits\_t} which defines an
inner type \code{type\_t} -- the \cplusplus\ counterpart of the \pascal\ subrange
type of the table above. The integral constants needed for these defintions are
taken in by \code{types\_traits\_t} from the class namespace of its template
argument \code{T}. Further to be seen below is that the template argument \code{T}
will actually name the \code{HashTrie<>} class itself (see the source lines
\ref{cpp-inner-types-traits-begin}\~\ref{cpp-inner-types-traits-end}
below: there is an instance of the well-known \cplusplus\ idiom of name
\emph{Curiously Recurring Template Pattern}, \emph{CRTP}).
%
% $ hash-trie-source --types-traits --tex -t -r 1710-1714 -r 1731-1735 -r 1752-1756 -r 1773-1777
%
\begin{hashtrielisting}[1696]
template<typename T>
struct types_traits_t<T, count_type>
{
	typedef unsigned short base_t;
#if defined(CONFIG_HASH_TRIE_STRICT_TYPES)
	typedef Ext::box_t<
		base_t, types_traits_t<T, count_type>> type_t;
#else
	typedef base_t type_t;
#endif

	// count_t: [0..max_count]
	static constexpr size_t min = 0;
	static constexpr size_t max = T::max_count;@\skipnumbering{1715}@
};

template<typename T>
struct types_traits_t<T, pointer_type>
{
	typedef unsigned short base_t;
#if defined(CONFIG_HASH_TRIE_STRICT_TYPES)
	typedef Ext::box_t<
		base_t, types_traits_t<T, pointer_type>> type_t;
#else
	typedef base_t type_t;
#endif

	// pointer_t: [0..trie_size]
	static constexpr size_t min = 0;
	static constexpr size_t max = T::trie_size;@\skipnumbering{1736}@
};

template<typename T>
struct types_traits_t<T, letter_type>
{
	typedef unsigned char base_t;
#if defined(CONFIG_HASH_TRIE_STRICT_TYPES)
	typedef Ext::box_t<
		base_t, types_traits_t<T, letter_type>> type_t;
#else
	typedef base_t type_t;
#endif

	// letter_t: [min_letter..max_letter]
	static constexpr size_t min = T::min_letter;
	static constexpr size_t max = T::max_letter;@\skipnumbering{1757}@
};

template<typename T>
struct types_traits_t<T, cell_type>
{
	typedef unsigned char base_t;
#if defined(CONFIG_HASH_TRIE_STRICT_TYPES)
	typedef Ext::box_t<
		base_t, types_traits_t<T, cell_type>> type_t;
#else
	typedef base_t type_t;
#endif

	// cell_t: [empty_slot..header]
	static constexpr size_t min = T::empty_slot;
	static constexpr size_t max = T::header;@\skipnumbering{1778}@
};
\end{hashtrielisting}
%
The purpose of template \code{char\_traits\_t} is that of alleviating the
cumbersome \pascal-specific mechanisms embodied by the tables
\code{lowercase},
\code{uppercase} and
\code{lettercode}
of Knuth's implementation
(see the accounts on \pagerefrange{pascal-lettercode-begin}{pascal-lettercode-end}).
Designing \code{HashTrie<>} with a template template parameter \code{T}
framing up the \emph{character representation traits} of the class was a
consequence of the following requirement: the class need to avoid
hard-coding its dependecies on character representation peculiarities within
the implementation code itself. Having the template parameter \code{T} --
with the default value set to the fully fledged \code{char\_traits\_t} \structtempl\ --
enhances the flexibility of the implementation of \code{HashTrie<>} 
working transparently with several different types of character representations.
%
% $ hash-trie-source --char-traits --tex -t -r 1921-1929 -r 1937-1942 -r 1944-1947 -r 1949-1952 -r 1954-1966 -r 1968-1976 -r 1979-1980
%
\begin{hashtrielisting}[1899]
template<typename C = char>
struct char_traits_t
{
	typedef
		typename
			std::remove_cv<C>::type
		char_t;
	typedef
		Ext::unsigned_t<char_t>
		uchar_t;
	typedef
		std::char_traits<char_t>
		traits_t;
	typedef
		Ext::unsigned_t<
			typename traits_t::int_type>
		uint_t;
	typedef
		size_t code_t;
	typedef
		std::basic_string<char_t>
		string_t;@\skipnumbering{1930}@
	static constexpr uint_t to_int(char_t ch)
	{ return traits_t::to_int_type(ch); }

	static constexpr char_t to_char(uint_t uint)
	{ return traits_t::to_char_type(uint); }

	static constexpr char_t to_lower(char_t ch);@\skipnumbering{1943}@
	static constexpr bool is_valid_char(char_t a)@\skipnumbering{1948}@
	static constexpr bool is_valid_code(code_t a)@\skipnumbering{1953}@
	static code_t to_code(char_t c)@\skipnumbering{1967}@
	static char_t from_code(code_t c)@\skipnumbering{1977}@
	static constexpr code_t min_letter = 1;
	static constexpr code_t max_letter = min_letter +@\skipnumbering{1981}@
	static constexpr code_t header     = max_letter + 1;
	static constexpr code_t empty_slot = 0;
};
\end{hashtrielisting}
%
The sole role of \code{size\_traits\_t} is that of encapsulating the \pascal\
\code{define}s on \pagerefrange{pascal-types-begin}{pascal-types-end}:
%
% $ hash-trie-source --size-traits --tex -t
%
\begin{hashtrielisting}[1999]
struct size_traits_t
{
	static constexpr size_t max_count = 32767;
	static constexpr size_t trie_size = CONFIG_HASH_TRIE_TRIE_SIZE; // = 32767;
	static constexpr size_t tolerance = CONFIG_HASH_TRIE_TOLERANCE; // = 1000;
};
\end{hashtrielisting}
%
The \classtempl\ \code{HashTrie<>} has three template arguments: \code{C},
\code{T} and \code{S}. \code{C} is the type used by the class for character
representation; \code{T} is the traits class associated to the character
representation type of the class; \code{S} is the class of which namespace
defines statically three size-related constants parametrizing \code{HashTrie<>}.
Sensible defaults were provided for each of these the template parameters --
defaults which were already presented above.

The public interface of \code{HashTrie<>} is quite simple: a default constructor,
two overloaded methods \code{put}, and three printing related methods:
\code{print},
\code{dump} and
\code{print\_stats}.
The third of these printing methods is available only when \code{hash-trie.cpp}
was compiled with the \emph{statistics collecting code} enabled, \ie\ when the
\code{CONFIG\_HASH\_TRIE\_STATISTICS} configuration parameter was
\define d.
%
% $ hash-trie-source --hash-trie --tex -t -r 2035-2037 -r 2048-2105 -r 2120-2199 -r 2201-2205 -r 2232-2239 -r 2245-2248 -r 2256-2258 -r 2260-2265 -r 2270-2280 -r 2290-2291 -l 2038:cpp-inner-types-traits-begin -l 2047:cpp-inner-types-traits-end -l 2210:cpp-types-first-line -l 2228:cpp-types-last-line -l 2259:cpp-make-alpha-expr
%
\begin{hashtrielisting}[2006]
template<
	typename C = char,
	template<typename> class T = char_traits_t,
	typename S = size_traits_t>
class HashTrie :
	private T<C>,
	private S
{
public:
	typedef S size_traits_t;
	typedef T<C> char_traits_t;
	typedef typename char_traits_t::char_t char_t;
	typedef typename char_traits_t::string_t string_t;

	HashTrie();

	bool put(const char_t*);

	bool put(const string_t& str)
	{ return put(str.c_str()); }

	void print(std::basic_ostream<char_t>&) const;
	void dump(std::basic_ostream<char_t>&) const;

#ifdef CONFIG_HASH_TRIE_STATISTICS
	void print_stats(std::basic_ostream<char_t>&) const;
#endif

private:@\skipnumbering{2038}@
	template<types_t t>@\label{cpp-inner-types-traits-begin}@
	using types_traits_t = types_traits_t<HashTrie, t>;

	using size_traits_t::max_count;
	using size_traits_t::trie_size;
	using size_traits_t::tolerance;
	using char_traits_t::min_letter;
	using char_traits_t::max_letter;
	using char_traits_t::empty_slot;
	using char_traits_t::header;@\label{cpp-inner-types-traits-end}@@\skipnumbering{2106}@
	template<types_t i, types_t e>
	struct array_of_t
	{
		typedef
			typename types_traits_t<e>::type_t
			elem_t;

		class type_t
		{
		public:
			static constexpr size_t lo =
				types_traits_t<i>::min;
			static constexpr size_t hi =
				types_traits_t<i>::max;@\skipnumbering{2200}@
		private:@\skipnumbering{2206}@
			elem_t array[hi - lo + 1];
		};
	};

	typedef typename types_traits_t<count_type>::type_t   count_t;@\label{cpp-types-first-line}@
	typedef typename types_traits_t<pointer_type>::type_t pointer_t;
	typedef typename types_traits_t<letter_type>::type_t  letter_t;
	typedef typename types_traits_t<cell_type>::type_t    cell_t;

	typedef
		typename array_of_t<pointer_type, count_type>::type_t
		count_array_t;
	typedef
		typename array_of_t<pointer_type, pointer_type>::type_t
		pointer_array_t;
	typedef
		typename array_of_t<pointer_type, cell_type>::type_t
		cell_array_t;

	cell_array_t    ch;
	count_array_t   count;
	pointer_array_t link;
	pointer_array_t sibling;@\label{cpp-types-last-line}@

	// x_n = (alpha * n) % mod_x
	pointer_t x;@\skipnumbering{2240}@
	static letter_t to_letter(char_t c)
	{ return char_traits_t::to_code(c); }

	static char_t to_char(letter_t l)
	{ return char_traits_t::from_code(l); }  @\skipnumbering{2249}@
	pointer_t find(const char_t*);

	void print_word(pointer_t, std::basic_ostream<char_t>&, bool) const;
	void dump_word(pointer_t, std::basic_ostream<char_t>&, bool&) const;
	void dump_empty(std::basic_ostream<char_t>&) const;

	static constexpr size_t make_alpha(size_t trie_size, size_t max_letter)@\skipnumbering{2259}@
	{ return std::ceil(0.61803 * (trie_size - 2 * max_letter)); }@\label{cpp-make-alpha-expr}@@\skipnumbering{2266}@
	static constexpr size_t alpha = make_alpha(trie_size, max_letter);
	static constexpr size_t mod_x = trie_size - 2 * max_letter;
	static constexpr size_t max_h = mod_x + max_letter;
	static constexpr size_t max_x = mod_x - alpha;@\skipnumbering{2281}@
#ifdef CONFIG_HASH_TRIE_STATISTICS
	struct stats_t
	{
		size_t loaded_words = 0;
		size_t failed_words = 0;
		size_t duplicated_words = 0;
		size_t empty_slots = 0;

		Sys::clocks_t trie_time = {};@\skipnumbering{2292}@
	};

	mutable stats_t stats;
#endif
};
\end{hashtrielisting}
%
The \structtempl\ \code{HashTrie<>::array\_of\_t<>} and its inner defined class
\code{type\_t} is very important to the internal economy of class \code{HashTrie<>}.
This is so because the class \code{array\_of\_t<>::type\_t} is implementing
the \cplusplus\ counterparts of the \pascal\ \code{array}s seen on  
\pagerefrange{pascal-types-begin}{pascal-types-end}. This class provides two
oveloaded methods (which were cut off from the listing above), namely
\code{operator[]} and \code{assign}. Each of these methods are to be seen
at work in the constructor of class \code{HashTrie<>} and in its \code{find}
method.

The method templates \code{operator[]} are offering transparently
range-checked or range-unchecked access to an underlying \cplain-style array.
The dimension of this \cplain-array is determined by the first template argument
of \code{array\_of\_t<>}, while the second template argument of \code{array\_of\_t<>}
decides the type of the elements of this array. Yet an important feature of the
\code{operator[]} methods of \code{array\_of\_t<>::type\_t} is the following:
these methods model the semantics of \pascal\ \code{array}s -- that is that
indexing within the structure doesn't start at \code{0}, but at the position
defined by the respective type declaration.

A third significant trait of these \code{operator[]} methods is that they
accept transparently boxed or just plain integer typed values as indices.
\code{HashTrie<>} uses boxed integers within its internal workings if and
only if the configuration parameter \code{CONFIG\_HASH\_TRIE\_STRICT\_TYPES}
is \define d at the time of its compilation.

The alternative of checking or not the indexed access into such an \emph{array}
structure is arbitrated by the configuration parameter
\code{CONFIG\_HASH\_TRIE\_ARRAY\_BOUNDS}: indexing is range-checked if and only
if this parameter is \define d at the time of compilation of \code{hash-trie.cpp}.

The methods
\code{``void assign(size\_t first, size\_t last, \emph{type} value)''}
do assign \code{value} to each element of the underlying array in
the range \code{[first, last]}. Similarly to the \code{operator[]} methods,
these \code{assign} methods do range checking if and only if the configuration
parameter \code{CONFIG\_HASH\_TRIE\_ARRAY\_BOUNDS} was \define d prior to
initiate the compilation of \code{hash-trie.cpp}.

At this moment, one can easily examine the veracity of the forecast made in
subsection \ref{knuth-pascal-impl} about the core \pascal\ structures of \hashtrie*\
defined on \pagerefrange{pascal-types-begin}{pascal-types-end} to be tightly
parallel with the corresponding \cplusplus\ structures. The definitions of
\code{link},
\code{sibling},
\code{ch} and
\code{count},
as shown by the \pascal\ source lines
\ref{pas-types-first-line}\~\ref{pas-types-last-line} on
\pagerefrange{pas-types-first-line}{pas-types-last-line},
and their parallel \cplusplus\ definitions shown by source lines
\ref{cpp-types-first-line}\~\ref{cpp-types-last-line} on
\pagerefrange{cpp-types-first-line}{cpp-types-last-line},
are identical indeed.

One final note about the declaration of \code{HashTrie<>} concerns its static
function \code{make\_alpha}: even though \code{make\_alpha} makes use of
the standard non-\code{constexpr} function \code{std::ceil}, its declaration as
\code{constexpr} is, nota bene!, correct.\footnote{%
By the \cplusplus11 standard, ISO/IEC 14882:2011,
the function \code{std::ceil} doesn't have to be \code{constexpr}, yet
\prog{gcc} of version at least \code{4.8.0}, in the header file \code{cmath},
boils \code{std::ceil} down to the undocumented built-ins
\code{\_\_builtin\_ceilf} and 
\code{\_\_builtin\_ceill}.}

The constructor of class \code{HashTrie<>} is very simple: it initializes the
inner structures and variables of the class exactly as Knuth does 
\cite[\parapg{19}{159} and \parapg{23}{160}]{knuth:literate}:
%
% $ hash-trie-source --hash-trie-func=HashTrie --tex -t -l 2314:cpp-init-count-zero
%
\begin{hashtrielisting}[2298]
template<
	typename C,
	template<typename> class T,
	typename S>
HashTrie<C, T, S>::HashTrie()
{
	ch.assign(header, trie_size, empty_slot);
	link.assign(1, max_letter, 0);
	count.assign(1, max_letter, 0);

	for (pointer_t i = 1; i <= max_letter; ++ i) {
		ch[i] = i;
		sibling[i] = i - 1;
	}

	link[0] = 0;
	count[0] = 0;@\label{cpp-init-count-zero}@
	ch[0] = header;
	sibling[0] = header - 1;

	x = 0;
}
\end{hashtrielisting}
%
The \code{put} methods of \code{HashTrie<>} are little more than just public
front-ends to the hard-working private function \code{find}. Nevertheless,
the \code{put} method shown below does certain housekeeping:
it checks for valid user input\footnote{The \emph{trie} structures
%!!!\cite[\parapg{6.3}{492}]{knuth:taocp3}
\cite[\para{6.3}]{knuth:taocp3}
-- in particular the \emph{hash-tries} --, by definition, cannot
contain empty words.}, increments the counter associated to the input word,
if any, and, when \code{CONFIG\_HASH\_TRIE\_STATISTICS}
is \define d, updates its \code{stats} structure:
%
% $ hash-trie-source --hash-trie-func=put --tex -t
%
\begin{hashtrielisting}[2644]
template<
	typename C,
	template<typename> class T,
	typename S>
bool HashTrie<C, T, S>::put(const char_t* str)
{
	if (*str == '\0')
		Sys::error<Error>("hash tries cannot contain empty words");

#ifdef CONFIG_HASH_TRIE_STATISTICS
	Sys::utime_t time;
#endif

	// 34. Input the text, maintaining a dictionary with frequency count
	auto p = find(str);

#ifdef CONFIG_HASH_TRIE_STATISTICS
	stats.trie_time += time();

	if (p)
		stats.loaded_words ++;
	else
		stats.failed_words ++;
	if (p && count[p])
		stats.duplicated_words ++;
#endif

	if (!p) return false;
	count[p] ++;
	return true;
}
\end{hashtrielisting}
%
The method \code{find} of \code{HashTrie<>} class constitutes the central part
of Knuth's \hashtrie*\ data structure. It substantiates the core algoritm of this
data structure -- it is the \cplusplus\ counterpart of Knuth's \pascal\ function
\code{find\_buffer}
\cite[\paras{20}{21} and \paraspp{24}{31}{159}{162}]{knuth:literate}:
%
% $ hash-trie-source --hash-trie-func=find --tex -t -r 2329-2444 -r 2453-2476 -r 2483-2505 -r 2507-2522 -r 2526-2532 -r 2539-2544 -r 2565 -r 2569 -r 2571 -r 2600 -r 2604 -r 2614 -l 2449:cpp-last-h-def -l 2594:cpp-delta-def-int -l 2523:cpp-last-h-assign-1 -l 2533:cpp-last-h-assign-2 -l 2607:cpp-delta-assign-1 -l 2617:cpp-delta-assign-2
%
\begin{hashtrielisting}[2321]
template<
	typename C,
	template<typename> class T,
	typename S>
typename
	HashTrie<C, T, S>::pointer_t
	HashTrie<C, T, S>::find(const char_t* str)
{@\skipnumbering{2445}@
	const pointer_t tolerance2 = tolerance;
	// trial header location
	pointer_t h;
	// the final one to try
	pointer_t last_h; //!!!INT int last_h;@\label{cpp-last-h-def}@

	const auto get_set_for_computing_header_locations = [&]() {
		// 24. Get set for computing header locations@\skipnumbering{2477}@
		if (x >= max_x)
			x -= max_x;
		else
			x += alpha;

		h = x + max_letter + 1; // now max_letter < h <= trie_size - max_letter@\skipnumbering{2506}@
		if (h > max_h - tolerance) {@\skipnumbering{2523}@
			last_h = add(h, tolerance2) - mod_x;@\label{cpp-last-h-assign-1}@
		}
		else {@\skipnumbering{2533}@
			last_h = h + tolerance;@\label{cpp-last-h-assign-2}@
		}
	};

	const auto compute_the_next_trial_header_location = [&]() {
		// 25. Compute the next trial header location h, or abort find@\skipnumbering{2545}@
		if (h == last_h)
			return false;
		if (h == max_h)
			h = max_letter + 1;
		else
			h ++;
		return true;
	};

	// the current word position
	pointer_t p
		= to_letter(*str ++);
	while (*str) {
		// current letter code
		letter_t c
			= to_letter(*str ++);
		// 21. Advance p to its child number c
		if (link[p] == 0) {
			// 27. Insert the firstborn child of p and move to it, or abort find
			get_set_for_computing_header_locations();@\skipnumbering{2566}@
			do {
				if (!compute_the_next_trial_header_location())
					return 0;@\skipnumbering{2570}@
			} while (ch[h] != empty_slot || ch[h + c] != empty_slot);@\skipnumbering{2572}@
			link[p] = h;
			link[h] = p;
			p = h + c;
			ch[h] = header;
			ch[p] = c;
			sibling[h] = p;
			sibling[p] = h;
			count[h] = 0;
			count[p] = 0;
			link[p] = 0;
		}
		else {
			// the next word position
			pointer_t q
				= link[p] + c;
			if (ch[q] != c) {
				if (ch[q] != empty_slot) {
					// 29. Move p's family to a place where child c will fit,
					//     or abort find
					// family member to be moved
					pointer_t r;
					// amount of motion
					//!!!INT int delta;@\label{cpp-delta-def-int}@
					// have we found a new homestead?
					bool slot_found;
					// 31. Find a suitable place h to move, or abort find
					slot_found = false;
					get_set_for_computing_header_locations();@\skipnumbering{2601}@
					do {
						if (!compute_the_next_trial_header_location())
							return 0;@\skipnumbering{2605}@
						if (ch[h + c] == empty_slot) {
							r = link[p];
							auto delta = sub(h, r); //!!!INT delta = h - r;@\label{cpp-delta-assign-1}@
							while (ch[r + delta] == empty_slot &&
									sibling[r] != link[p])
								r = sibling[r];
							slot_found = ch[r + delta] == empty_slot;
						}
					} while (!slot_found);@\skipnumbering{2615}@
					q = h + c;
					r = link[p];
					auto delta = sub(h, r); //!!!INT delta = h - r;@\label{cpp-delta-assign-2}@
					do {
						sibling[r + delta] = sibling[r] + delta;
						ch[r + delta] = ch[r];
						ch[r] = empty_slot;
						count[r + delta] = count[r];
						link[r + delta] = link[r];
						if (link[r]) link[link[r]] = r + delta;
						r = sibling[r];
					} while (ch[r] != empty_slot);
				}
				// 28. Insert child c into p's family
				h = link[p];
				while (sibling[h] > q)
					h = sibling[h];
				sibling[q] = sibling[h];
				sibling[h] = q;
				ch[q] = c;
				count[q] = 0;
				link[q] = 0;
			}
			p = q;
		}
	}
	return p;
}
\end{hashtrielisting}
%
Due to the infrastructure already built, the above implementation is almost
one hunderd percent a direct translation of Knuth's \pascal\ code to \cplusplus.
The few minute yet important differences between the two implementations are
listed by the table below. The second column of the table contains references
relative to \cite{knuth:literate} and the third column -- references to the
\cplusplus\ source code from above.
%
\begin{dtabular}
declaration of \code{last\_h} & \pgparalno{161}{26}{3} & \pgreflno{cpp-last-h-def}
\desc{The \pascal\ code declares the type of \code{last\_h} to be \code{integer}.
The \cplusplus\ code declares this variable to be of type \code{pointer\_t}.}
\erow

assignment to \code{last\_h} & \pgparalno{161}{24}{7} & \pgreflno{cpp-last-h-assign-1}
\desc{The \pascal\ code assigns to \code{last\_h} by the expression
``\code{h + tolerance - mod\_x}''. The \cplusplus\ code does
that by the expression ``\cppinline{add(h, tolerance2) - mod\_x}''.}
\erow

declaration of \code{delta} & \pgparalno{162}{30}{3} & \pgreflno{cpp-delta-def-int}
\desc{The \pascal\ code declares \code{delta} an \code{integer} and assigns
to it twice. The \cplusplus\ code removes the shared declaration and replaces
each assignment with a complete definition.}
\erow

1st assignment to \code{delta} & \pgparalno{162}{31}{5} & \pgreflno{cpp-delta-assign-1}
\desc{The \pascal\ code assigns to \code{delta} by the statement
``\code{delta = h - r}''. The \cplusplus\ code replaces this assignment with
``\cppinline{auto delta = sub(h, r)}''.}
\erow

2nd assignment to \code{delta} & \pgparalno{162}{29}{3} & \pgreflno{cpp-delta-assign-2}
\desc{The same difference as in the case of the 1st assignment to \code{delta}.}
\end{dtabular}
%
The difference of the two implementations stems from the fact that the
\code{Ext::box\_t<>} class template, which accomplishes the boxing of the integral
types used by \code{HashTrie<>} -- \ie\ the integer types of name \code{base\_t}
defined inside the class namespaces of each of the four instances of the class
template \code{types\_traits\_t<>} above --, is quite strict with regards
to the operations applied to values of its type. The class template
\code{Ext::box\_t<>} is supposed to simulate in \cplusplus\ the semantics
of subrange types of \pascal, that is that it has to do bounds checking
on assignments on values of its type. \code{Ext::box\_t<>} is, of course,
doing that and, more so, it checks the bounds on each of its arithmetical
operations too.

The replacement of expression ``\code{h + tolerance - mod\_x}''  with the
expression ``\code{add(h, tolerance2) - mod\_x}'' had to be done, because,
as it is easily provable, under cetain conditions, the subexpression
``\code{h + tolerance}'' is exceeding the upper bound \code{trie\_size}
of the boxed integer \code{pointer\_t}. It is worthy of notice the readily
provable fact that both assignments to \code{last\_h} on lines
\ref{cpp-last-h-assign-1} and \ref{cpp-last-h-assign-2} are correct:
each of the expression on the right side of these assignments do not
exceed upon evaluation the bounds of \code{pointer\_t}.

The two assignments to \code{delta} shown above had to be adjusted because of
two reasons. The first one is that upon subtracting two \code{pointer\_t}s
\code{h} and \code{r}, the result would be of type \code{pointer\_t} too.
However, the original intent of \code{delta} included the possibility of it
being a negative quantity. 

The other reason for the code adjustement applied to \code{delta} is obvious
once noticing that the boxed integral types used by \code{HashTrie<>} are each
parametrized. Thus one cannot hard-code the type of \code{delta} without
eventually getting into trouble. The function template \code{Ext::sub<>}
is designed such a way that it decides the type of its resulting value based
on the type of its arguments: the resulting type is choosen such that to avoid
the occurence of undeflow errors upon the subtraction of the two input
values.
%
\section{The \hashtrie\ Binary}
%
The \HashTrie's main component is the binary program \hashtrie\ which
results upon compiling and linking the source file \code{hash-trie.cpp}.
%
The command line options of \hashtrie\ program are as follows:
%
\begin{shelllisting}
$ ./hash-trie --help
usage: hash-trie [OPTION]...
where the options are:
  -P|--[print-]config  action: print out the config parameters
  -T|--[print-]types   action: print out the add/sub types table
  -L|--load-only       action: only load the input into the hash trie
  -p|--print[-trie]    action: print out all <word, count> pairs from
                         the hash trie (default)
  -d|--dump[-trie]     action: dump out the complete hash trie structure
  -b|--[no-][dump-]backtrace
                       dump a backtrace of the program on fatal error
                         or otherwise do not (default not)
  -c|--[no-][print-]chars
                       print out characters instead of codes on structure
                         dumps or otherwise do not (default not)
  -w|--[no-][print-]words
                       print out words when dumping out the hash trie
                         or otherwise do not (default do)
  -s|--[no-][print-]stats
                       print out some statistics information
                         or otherwise do not (default not)
     --time-type=TYPE  use the specified time type when printing out
                         timings; TYPE can be one of: real, user or sys;
                         the default is real
     --debug=WHAT      print out some debugging information; WHAT can
                         be one of: probing
     --no-debug        do not print debugging info at all (default)
  -q|--[no-]quiet      be quiet or otherwise do not (default not)
  -v|--version         print version numbers and exit
  -?|--help            display this help info and exit
\end{shelllisting}
%
The \hashtrie\ program takes input from standard input \code{stdin} -- when its
action option is one of
\opt{-L|--load-only},
\opt{-p|--print-trie} or
\opt{-d|--dump-trie}.
When the action option is either
\opt{-P|--print-config} or
\opt{-T|--print-types},
\hashtrie\ is not taking input at all -- it is only printing out something as
requested.
If input is to be read in, it is supposed to be formed by words -- \ie\ 
non-empty sequences of characters in range \code{[a-zA-Z]} -- each placed on a
separate line.

\label{hash-trie-config-params-begin}
As already known from the previous section, any given build of \hashtrie\ is
governed by the following six configuration parameters:
%
\begin{shelllisting}
$ grep -E -how 'CONFIG_[A-Z0-9_]+' hash-trie.cpp|sort -u
CONFIG_HASH_TRIE_ARRAY_BOUNDS
CONFIG_HASH_TRIE_CHAR_TYPE
CONFIG_HASH_TRIE_STATISTICS
CONFIG_HASH_TRIE_STRICT_TYPES
CONFIG_HASH_TRIE_TOLERANCE
CONFIG_HASH_TRIE_TRIE_SIZE
\end{shelllisting}
%
Also worthy of notice is that there exists the seventh governing parameter of
\hashtrie: namely \code{DEBUG}, which is defined by default and is not defined
when compiling \code{hash-trie.cpp} with code optimizations enabled. The \prog{gcc}
compiler is told to enable the level \code{$NUM$} of code optimizations when invoking
\prog{make} with option \opt{OPT=$NUM$}.
The options \opt{--debug=WHAT} and \opt{--no-debug} of \hashtrie\ are available
for usage only when \code{DEBUG} was defined at the time of compilation of the
program.

As a result of this condition, is easy to establish whether a given
\hashtrie\ binary was build with debugging code enabled or not: the count obtained
from the \prog{grep} command below is non-zero if and only if \code{DEBUG}
is enabled:
%
\begin{shelllisting}
$ make -B hash-trie OPT=0
gcc -std=gnu++11 -Wall -Wextra -DPROGRAM=hash-trie -O0 hash-trie.cpp -o hash-trie -lstdc++
$ ./hash-trie -?|grep -c debug
0
$
\end{shelllisting}
%
All of the above \code{CONFIG\_HASH\_TRIE\_*} parameters are orthogonal
to each other, with only two exceptions:
%
\begin{bitemize}
\item \code{STRICT\_TYPES} implies \code{ARRAY\_BOUNDS};
\item \code{TOLERANCE} and \code{TRIE\_SIZE}
are constraining each other by the internal logic of class \code{HashTrie<>}
in the source file \code{hash-trie.cpp}. (Look-up the source code for
\compiletime\ assertions \code{CXX\_ASSERT} applied to expressions containing
or relating to the static constants \code{tolerance} and \code{trie\_size}.)
\end{bitemize}
%
The two parameters \code{STRICT\_TYPES} and \code{ARRAY\_BOUNDS} imply certain
\runtime\ supplemental checks on the types involved in the class \code{HashTrie<>},
each of these checks being made of \runtime\ assertions of \code{SYS\_ASSERT} kind.
\code{SYS\_ASSERT} is \cplusplus\ preprocessing macro, which, however,
is ruled out when the \code{DEBUG} parameter is not defined. Consequently,
not defining \code{DEBUG} voids out all the supplemental restrictions
the \code{STRICT\_TYPES} and \code{ARRAY\_BOUNDS} parameters are enforcing.
There is also a positive side of this situation of \code{DEBUG} not being defined:
the \hashtrie\ program will improve significantly its runtime speed.

If \hashtrie\ was built with all its configuration parameters having their
default values, it would produce the following output when invoked for the
action option \opt{-P|--print-config}:
%
\begin{shelllisting}
$ make -B hash-trie
gcc -std=gnu++11 -Wall -Wextra -DDEBUG -DPROGRAM=hash-trie -g -gdwarf-3 hash-trie.cpp -o hash-trie -lstdc++
$ ./hash-trie -P
TOLERANCE:    1000
TRIE_SIZE:    32767
CHAR_TYPE:    char
STRICT_TYPES: no
ARRAY_BOUNDS: no
STATISTICS:   no
$
\end{shelllisting}
\label{hash-trie-config-params-end}
%
On a \gnulinux\ 64-bit Intel x86 CPU machine, the output of \hashtrie\ for action
option \opt{-T|--print-types} would look like:
%
% $ ./hash-trie -T|table-view -P
%
\begin{shelllisting}
$ ./hash-trie -T
op   operand         operator      wider               result
add  char            int           int                 int
add  signed char     int           int                 int
add  unsigned char   int           int                 int
add  short           int           int                 int
add  unsigned short  int           int                 int
add  int             int           long long           long long
add  unsigned int    unsigned int  unsigned long long  long long
sub  char            int           int                 int
sub  signed char     int           int                 int
sub  unsigned char   int           int                 int
sub  short           int           int                 int
sub  unsigned short  int           int                 int
sub  int             int           long long           long long
sub  unsigned int    unsigned int  unsigned long long  long long
\end{shelllisting}
%
Feeding \hashtrie\ in with two words \code{a} and \code{ab}, it would print out
the following text tables when the action option is \opt{-p|--print-trie} and,
respectively, \opt{-d|--dump-trie}:
%
% $ command -P make -B hash-trie
% $ print a ab|./hash-trie -p|ssed -R 's/\t/   /;s/^(.{3})\s*/\1/'
% $ print a ab|./hash-trie -d|head-tail -h4 -t7|ssed -R '/^\.\.\.$/{s//\x0\t\x0\t\x0\t\x0\t\x0\t\x0/;s/\x0/.../g}'|table-view -P|ssed -R 's/-//'
%
\begin{shelllisting}
$ print() { printf '%s\n' "$@"; }
$ print a ab|./hash-trie -p
a   1
ab  1
$ print a ab|./hash-trie -d
p      link[p]  ch[p]   sibling[p]  count[p]  word
0      0        header  26          0         
1      20247    1       0           1         a
2      0        2       1           0         b
...    ...      ...     ...         ...       ...
25     0        25      24          0         y
26     0        26      25          0         z
...    .        .       .           .         .
20247  1        header  20249       0         
...    .        .       .           .         .
20249  0        2       20247       1         ab
...    .        .       .           .         .
$
\end{shelllisting}
%
When invoked with action option \opt{-L|--load-only}, \hashtrie\ would suppress
its normal output, printing out only error messages. This behaviour is useful
for example in conjunction with options \opt{-s|--print-stats} and \opt{-q|--quiet}.
The options \opt{-s|--print-stats}, \opt{--no-print-stats} and \opt{--time-type}
are available to the user of \hashtrie\ only when the program was built with the
\code{STATISTICS} configuration parameter enabled (as in the \prog{make}
invocation below).
%
\begin{shelllisting}
$ make -B hash-trie CONFIG+=TOLERANCE=1 CONFIG+=TRIE_SIZE=55 CONFIG+=STATISTICS
gcc -std=gnu++11 -Wall -Wextra -DDEBUG -DCONFIG_HASH_TRIE_TOLERANCE=1 -DCONFIG_HASH_TRIE_TRIE_SIZE=55 -DCONFIG_HASH_TRIE_STATISTICS -DPROGRAM=hash-trie -g -gdwarf-3 hash-trie.cpp -o hash-trie -lstdc++
$ print|./hash-trie -L
hash-trie: error:1: hash tries cannot contain empty words
$ print _|./hash-trie -L
hash-trie: error:1: invalid input char '\x5f'
$ print a ab ac acd|./hash-trie -L
hash-trie: error:4: failed to put 'acd' in trie
$ print a ab ac acd|./hash-trie -L -s
hash-trie: error:4: failed to put 'acd' in trie
loaded-words:     3
failed-words:     1
duplicated-words: 0
empty-slots:      26
trie-time:        16
$ print a ab ac acd|./hash-trie -L -s -q
loaded-words:     3
failed-words:     1
duplicated-words: 0
empty-slots:      26
trie-time:        17
$
\end{shelllisting}
%
Note that the sole role of option \opt{-q|--quiet} is to inhibit the error messages
of type: \msgdot{hash-trie: error:LINE: failed to put 'WORD' in trie}
%
\begin{bbibliography}{Knuth00}
%
\bibitem[Kern81]{kernighan:pascal}
	Brian W. Kernighan. 1981.
	\emph{Why Pascal is Not My Favorite Programming Language}. 
	Technical Report 100, April, 1981.
	Computing Science, AT\&T Bell Laboratories,
	Murray Hill, NJ, USA.
	\url{http://www.lysator.liu.se/c/bwk-on-pascal.html}
\bibitem[Liang83]{liang:hyphenation}
	Franklin Mark Liang. 1983.
	\emph{Word hyphenation by computer}.
	Ph.D. Thesis, Technical Report STA\=CS\=83\=977.
	Stanford University, Department of Computer Science,
	Stanford, CA, USA.
	\url{http://www.tug.org/docs/liang/liang-thesis.pdf}
\bibitem[BKM86]{bentley:pearls}
	Jon Bentley, Donald E. Knuth and Doug McIlroy. 1986.
	Programming Pearls: a literate program.
	\emph{Communications of the Association of Computing Machinery}
	29, 6 (June 1986), 471-483.
\bibitem[Knuth92]{knuth:literate}
	Donald E. Knuth. 1992.
	\emph{Literate Programming}.
	CSLI Lecture Notes Number 27.
	Stanford University, Center for the Study of Language and Information,
	Stanford, CA, USA.
\bibitem[Knuth98]{knuth:taocp3}
	Donald E. Knuth. 1998.
	\emph{The Art of Computer Programming,  Volume 3: (2nd Ed.) Sorting and Searching}.
	Addison Wesley Longman Publishing Co., Inc.,
	Redwood City, CA, USA.
\bibitem[Strou13]{stroustrup:tc++pl4}
	Bjarne Stroustrup. 2013.
	\emph{The C++ Programming Language} (4th ed.).
	Addison-Wesley Professional.
%
\end{bbibliography}
%
\let\thefootnote\relax%
\footnotetext{%
\renewcommand{\Hashtrie}{Hash-Trie}%
Copyright \copyright\ 2016\hspace{1em}\textbf{\vargyas}

This file is part of \HashTrie.

\HashTrie\ is free software: you can redistribute it and/or modify
it under the terms of the GNU General Public License as published by
the Free Software Foundation, either version 3 of the License, or
(at your option) any later version.

\HashTrie\ is distributed in the hope that it will be useful,
but WITHOUT ANY WARRANTY; without even the implied warranty of
MERCHANTABILITY or FITNESS FOR A PARTICULAR PURPOSE.  See the
GNU General Public License for more details.

You should have received a copy of the GNU General Public License
along with \HashTrie.  If not, see \url{http://www.gnu.org/licenses/}.}
%
\end{document}


